\include{algcomp_folha_header}
\usepackage{hyperref}
\newcommand{\real}[1]{\underline{\mathbf{#1}}} 
\newcommand{\mod}{\mathrm{mod}\:}

\begin{document}
\writetitle{17.10.2011}{3}{2.5}


\nextsect

\sectitle{Polin�mios}

\begin{enumerate}
\item Define a seguinte classe {\it Polinomio} em C++:
\begin{verbatim} 
class Polinomio {

    vector<int> coeficientes;
    bool nulo;

  public:
    Polinomio();   // criar o polin�mio nulo
    Polinomio(vector<int>);  // criar um polin�mio a partir dum vector
      
    void Print();		 	// imprimir o polin�mio
    unsigned int Grau(); 		// o grau do polin�mio
    Polinomio escalar(int);	 	// multiplica��o escalar a um polin�mio
    Polinomio operator + (Polinomio);	// soma de polin�mios
    Polinomio operator - (Polinomio);	// diferen�a de polin�mios
    Polinomio operator * (Polinomio);	// multiplica��o de polin�mio usando Karatsuba
}
\end{verbatim}

\item Define uma classe {\it Mod7} em C++ para trabalhar com o corpo finito de $7$ elementos $\ZZ_7$.
\begin{verbatim} 
class Mod7 {

    int numero;

  public:
    Mod7();   // criar o n�mero 0
    Mod7(int);  // criar m�dulo a partir dum inteiro
      
    void Print();		 	// imprimir o m�dulo        
    Mod7 operator + (Mod7);	// soma de m�dulos
    Mod7 operator - (Mod7);	// diferen�a de m�dulos
    Mod7 operator * (Mod7);	// multiplica��o de m�dulos
    Mod7 operator / (Mod7);	// divis�o de m�dulo por um m�dulo n�o nulo
}
\end{verbatim}


\item Adapte a classe {\it Polinomio} no ponto $1$ para criar uma classe {\it $Polinomio\_Mod7$} de polin�mios em $\ZZ_7[x]$. 
Inclua as seguintes fun��es na classe para calcular o quociente e o resto de polin�mios $\ZZ_7[x]$.
\begin{verbatim} 
    Polinomio_Mod7 Polinomio_Mod7::operator / (Polinomio_Mod7);		// o quociente da divis�o
    Polinomio_Mod7 Polinomio_Mod7::operator % (Polinomio_Mod7);		// o resto da divis�o 
\end{verbatim}
Escreva uma fun��o para calcular o m�ximo divisor comum entre dois polin�mios em $\ZZ_7[x]$:
\begin{verbatim} 
    Polinomio_Mod7 mdc(Polinomio_Mod7, Polinomio_Mod7);
\end{verbatim}
\end{enumerate}




\end{document}