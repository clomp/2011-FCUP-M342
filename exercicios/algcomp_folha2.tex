\include{algcomp_folha_header}
\usepackage{hyperref}
\newcommand{\real}[1]{\underline{\mathbf{#1}}} 
\newcommand{\mod}{\mathrm{mod}\:}
\begin{document}
\writetitle{3.10.2011}{2}{1}


\nextsect

\sectitle{Dom�nios Euclideanos}

\begin{enumerate}
\item Encontre $x,y \in \ZZ$ tais que $\mathrm{mdc}(a,b)=xa+yb$ para
\begin{enumerate}
\item $a=12$, $b=52$;
\item $a=123$, $b=456$;
\item $a=431$, $b=17$;
\end{enumerate}
\item Encontre $f(x),g(x)\in \QQ[x]$ tais que $\mathrm{mdc}(a(x),b(x)) = f(x)a(x)+g(x)b(x)$ para
\begin{enumerate}
\item $a(x)=12x^2-3x+5$, $b(x)=2x-3$;
\item $a(x)=x^4-x^2+1$, $b(x)=x^2-1$;
\item $a(x)=x^3+x^2-2$, $b(x)=x^3-2$;
\end{enumerate}
\item Encontre os inversos das seguintes elementos
\begin{enumerate}
\item $14$ em $\ZZ_{33}$
\item $x^4-1$ em $\QQ[x]/\langle x^3-x^2-1\rangle$
\end{enumerate}
\item Reslova o seguinte sistema
$$\left\{ \begin{array}{rl} 
x &\equiv 12 (\mod 13) \\
x &\equiv 2 (\mod 7) \\
x &\equiv 14 (\mod 23) \end{array}\right.$$
\item Resolva a equia��o Diofantina linear
$$ 2x + 5y + 7z = 12$$
\end{enumerate}




\end{document}