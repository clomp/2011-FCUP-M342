\include{algcomp_folha_header}
\usepackage{hyperref}

\newcommand{\real}[1]{\underline{\mathbf{#1}}} 
\newcommand{\mod}{\mathrm{mod}\:}

\begin{document}
\writetitle{05.12.2011}{8}{8}


\nextsect

\sectitle{Algoritmo de divis�o}

\begin{enumerate}
\item Calcule o resto da divis�o de $f$ por $F$ com a ordem monomial indicado:
\begin{enumerate}
\item $f=x^7y^2+x^3y^2-y+1,\qquad F=(xy^2-x, x-y^3), \qquad >_{\mathrm{lex}} \mbox{  com  } x>y$
\item $f=x^7y^2+x^3y^2-y+1,\qquad F=(x-y^3, xy^2-x), \qquad >_{\mathrm{lex}} \mbox{  com  } x>y$
\item $f=x^7y^2+x^3y^2-y+1,\qquad F=(xy^2-x, x-y^3), \qquad >_{\mathrm{grlex}} \mbox{  com  } x>y$
\item $f=x^7y^2+x^3y^2-y+1,\qquad F=(x-y^3, xy^2-x), \qquad >_{\mathrm{grlex}} \mbox{  com  } x>y$
\item $f=xy^2z^2+xy-yz,\qquad F=(x-y^2, y-z^3, z^2-1), \qquad >_{\mathrm{lex}} \mbox{  com  } x>y>z$
\item $f=xy^2z^2+xy-yz,\qquad F=(y-z^3, z^2-1, x-y^2), \qquad >_{\mathrm{lex}} \mbox{  com  } x>y>z$
\item $f=xy^2z^2+xy-yz,\qquad F=(z^2-1, x-y^2, y-z^3), \qquad >_{\mathrm{lex}} \mbox{  com  } x>y>z$
\end{enumerate}
\item Sejam $f_1=x^2y-z$, $f_2=xy-1$ e $f=x^3-x^2y -x^2z+x$.
\begin{enumerate}
\item Calcule $r_1$, o resto da divis�o de $f$ por $F=(f_1,f_2)$ utilizando a ordem {\it lex}.
\item Calcule $r_2$, o resto da divis�o de $f$ por $F=(f_2,f_1)$ utilizando a ordem {\it lex}.
\item Seja $r=r_1-r_2$. Determine se $r\in I=\langle f_1,f_2\rangle$.
\item Sem calculo determine o resto da divis�o de $r$ por $F=(f_1,f_2)$.
\end{enumerate}
\item Sejam $V=V(y-x^2,z-x^3)$ e $h=z^2-x^4y$.
\begin{enumerate}
\item Mostre que $h(a,b,c)=0$ para todo ponto $(a,b,c)\in V$.
\item Utilizando uma ordem monomial adequada e divis�o de $h$ por $F=(y-x^2,z-x^3)$ para mostrar que $h\in \langle y-x^2,z-x^3\rangle$.
\end{enumerate}

\end{enumerate}




\end{document}
