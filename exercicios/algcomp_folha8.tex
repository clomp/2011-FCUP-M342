\documentclass{article}
\usepackage{a4wide}
\usepackage[latin1]{inputenc}
\usepackage{amssymb}

\newcommand{\nlim}[1]{\displaystyle{\lim_{n\rightarrow \infty} \: {#1}}}
\newcommand{\ser}{\sum_{n=1}^\infty \:}
\newcommand{\sern}[1]{{\sum_{n={#1}}^\infty\:}}
\newcommand{\dsum}{\displaystyle\sum}
\newcommand{\ZZ}{\mathbb{Z}}
\newcommand{\RR}{\mathbb{R}}
\newcommand{\NN}{\mathbb{N}}
\newcommand{\QQ}{\mathbb{Q}}
\newcommand{\CC}{\mathbb{C}}
\newcommand{\HH}{\mathbb{H}}
\newcommand{\mdc}{\mathrm{mdc}}
\newcommand{\arctg}{\mathrm{arc\:tg}}
\newcommand{\sen}[1]{\mathrm{sen}(#1)}
\renewcommand{\deg}[1]{\mathrm{grau}(#1)}
\newcommand{\Aut}[1]{\mathrm{Aut}(#1)}
\newcommand{\sgn}[1]{\mathrm{sgn}(#1)}
\newcommand{\MatrixZ}[4]{\left(\begin{array}{rr} #1 & #2 \\ #3 & #4 \end{array}\right)}
%\newcommand{\cos}{\mathrm{cos}}
\newcommand{\senh}{\mathrm{sh}}
%\newcommand{\cosh}{\mathrm{cosh}}
\newcommand{\len}{\mathrm{len}}
\newcommand{\arctgh}{\mathrm{arc\:tgh}}
%\newcounter{kapitel}
\newcounter{abschnitt}
\newcounter{uebung}
\newcommand{\reset}[1]{\setcounter{#1}{1}}
%\newcommand{\nextchap}{\addtocounter{kapitel}{1} \reset{abschnitt}\reset{uebung}}
\newcommand{\nextsect}{\addtocounter{abschnitt}{1} \reset{uebung} }
\newcommand{\nextexer}{\addtocounter{uebung}{1}}
%\newcommand{\getex}{(\arabic{kapitel}.\arabic{abschnitt}.\arabic{uebung})}
\newcommand{\getex}{(\arabic{abschnitt}.\arabic{uebung})}
\newcommand{\exitem}{\item[\getex] \nextexer}
\newcommand{\tpcexitem}{\item[\tpc \getex] \nextexer}
\renewcommand{\exp}[1]{\mathbf{e}^{#1}}
\newcommand{\sectitle}[1]{\begin{center}{\bf{\arabic{abschnitt}. \underline{#1}}}\end{center}}

%\setlength{\marginparwidth}{1.2in}
%\let\oldmarginpar\marginpar
%\renewcommand\marginpar[1]{\-\oldmarginpar[\raggedleft\footnotesize #1]%
%{\raggedright\footnotesize #1}}

\newcommand{\tpc}{{\bf $\mathbf{\star} \:\: $}}



%\newcommand{\writetitle}[6]{\bigskip \setcounter{kapitel}{#3}\setcounter{abschnitt}{#4}\reset{uebung}
\newcommand{\writetitle}[6]{\bigskip \setcounter{abschnitt}{#3}\reset{uebung}
\begin{tabular}{lp{6cm}r}
Faculdade de Ci�ncias da Universidade do Porto && Folha \bf{{#2}}\\
Departamento de Matem�tica&& #1\\
Christian Lomp&&
\end{tabular}
\bigskip\begin{center}
\bf{\Large{Exerc�cios de �lgebra Computacional (M342)}}\\[3mm] 
\end{center}\bigskip}

\pagestyle{empty}

%\flushright{ }
 \addtolength\topmargin{-2cm}
 \setlength{\textwidth}{17cm}
\addtolength\textheight{3cm} 
%\setlength{\textheight}{26cm}
 \setlength{\evensidemargin}{-5mm}
 \setlength{\oddsidemargin}{-5mm}

\usepackage{hyperref}

\newcommand{\real}[1]{\underline{\mathbf{#1}}} 
\newcommand{\mod}{\mathrm{mod}\:}

\begin{document}
\writetitle{05.12.2011}{8}{8}


\nextsect

\sectitle{Algoritmo de divis�o}

\begin{enumerate}
\item Calcule o resto da divis�o de $f$ por $F$ com a ordem monomial indicado:
\begin{enumerate}
\item $f=x^7y^2+x^3y^2-y+1,\qquad F=(xy^2-x, x-y^3), \qquad >_{\mathrm{lex}} \mbox{  com  } x>y$
\item $f=x^7y^2+x^3y^2-y+1,\qquad F=(x-y^3, xy^2-x), \qquad >_{\mathrm{lex}} \mbox{  com  } x>y$
\item $f=x^7y^2+x^3y^2-y+1,\qquad F=(xy^2-x, x-y^3), \qquad >_{\mathrm{grlex}} \mbox{  com  } x>y$
\item $f=x^7y^2+x^3y^2-y+1,\qquad F=(x-y^3, xy^2-x), \qquad >_{\mathrm{grlex}} \mbox{  com  } x>y$
\item $f=xy^2z^2+xy-yz,\qquad F=(x-y^2, y-z^3, z^2-1), \qquad >_{\mathrm{lex}} \mbox{  com  } x>y>z$
\item $f=xy^2z^2+xy-yz,\qquad F=(y-z^3, z^2-1, x-y^2), \qquad >_{\mathrm{lex}} \mbox{  com  } x>y>z$
\item $f=xy^2z^2+xy-yz,\qquad F=(z^2-1, x-y^2, y-z^3), \qquad >_{\mathrm{lex}} \mbox{  com  } x>y>z$
\end{enumerate}
\item Sejam $f_1=x^2y-z$, $f_2=xy-1$ e $f=x^3-x^2y -x^2z+x$.
\begin{enumerate}
\item Calcule $r_1$, o resto da divis�o de $f$ por $F=(f_1,f_2)$ utilizando a ordem {\it lex}.
\item Calcule $r_2$, o resto da divis�o de $f$ por $F=(f_2,f_1)$ utilizando a ordem {\it lex}.
\item Seja $r=r_1-r_2$. Determine se $r\in I=\langle f_1,f_2\rangle$.
\item Sem calculo determine o resto da divis�o de $r$ por $F=(f_1,f_2)$.
\end{enumerate}
\item Sejam $V=V(y-x^2,z-x^3)$ e $h=z^2-x^4y$.
\begin{enumerate}
\item Mostre que $h(a,b,c)=0$ para todo ponto $(a,b,c)\in V$.
\item Utilizando uma ordem monomial adequada e divis�o de $h$ por $F=(y-x^2,z-x^3)$ para mostrar que $h\in \langle y-x^2,z-x^3\rangle$.
\end{enumerate}

\end{enumerate}




\end{document}
