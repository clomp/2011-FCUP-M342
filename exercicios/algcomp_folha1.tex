\documentclass{article}
\usepackage{a4wide}
\usepackage[latin1]{inputenc}
\usepackage{amssymb}

\newcommand{\nlim}[1]{\displaystyle{\lim_{n\rightarrow \infty} \: {#1}}}
\newcommand{\ser}{\sum_{n=1}^\infty \:}
\newcommand{\sern}[1]{{\sum_{n={#1}}^\infty\:}}
\newcommand{\dsum}{\displaystyle\sum}
\newcommand{\ZZ}{\mathbb{Z}}
\newcommand{\RR}{\mathbb{R}}
\newcommand{\NN}{\mathbb{N}}
\newcommand{\QQ}{\mathbb{Q}}
\newcommand{\CC}{\mathbb{C}}
\newcommand{\HH}{\mathbb{H}}
\newcommand{\mdc}{\mathrm{mdc}}
\newcommand{\arctg}{\mathrm{arc\:tg}}
\newcommand{\sen}[1]{\mathrm{sen}(#1)}
\renewcommand{\deg}[1]{\mathrm{grau}(#1)}
\newcommand{\Aut}[1]{\mathrm{Aut}(#1)}
\newcommand{\sgn}[1]{\mathrm{sgn}(#1)}
\newcommand{\MatrixZ}[4]{\left(\begin{array}{rr} #1 & #2 \\ #3 & #4 \end{array}\right)}
%\newcommand{\cos}{\mathrm{cos}}
\newcommand{\senh}{\mathrm{sh}}
%\newcommand{\cosh}{\mathrm{cosh}}
\newcommand{\len}{\mathrm{len}}
\newcommand{\arctgh}{\mathrm{arc\:tgh}}
%\newcounter{kapitel}
\newcounter{abschnitt}
\newcounter{uebung}
\newcommand{\reset}[1]{\setcounter{#1}{1}}
%\newcommand{\nextchap}{\addtocounter{kapitel}{1} \reset{abschnitt}\reset{uebung}}
\newcommand{\nextsect}{\addtocounter{abschnitt}{1} \reset{uebung} }
\newcommand{\nextexer}{\addtocounter{uebung}{1}}
%\newcommand{\getex}{(\arabic{kapitel}.\arabic{abschnitt}.\arabic{uebung})}
\newcommand{\getex}{(\arabic{abschnitt}.\arabic{uebung})}
\newcommand{\exitem}{\item[\getex] \nextexer}
\newcommand{\tpcexitem}{\item[\tpc \getex] \nextexer}
\renewcommand{\exp}[1]{\mathbf{e}^{#1}}
\newcommand{\sectitle}[1]{\begin{center}{\bf{\arabic{abschnitt}. \underline{#1}}}\end{center}}

%\setlength{\marginparwidth}{1.2in}
%\let\oldmarginpar\marginpar
%\renewcommand\marginpar[1]{\-\oldmarginpar[\raggedleft\footnotesize #1]%
%{\raggedright\footnotesize #1}}

\newcommand{\tpc}{{\bf $\mathbf{\star} \:\: $}}



%\newcommand{\writetitle}[6]{\bigskip \setcounter{kapitel}{#3}\setcounter{abschnitt}{#4}\reset{uebung}
\newcommand{\writetitle}[6]{\bigskip \setcounter{abschnitt}{#3}\reset{uebung}
\begin{tabular}{lp{6cm}r}
Faculdade de Ci�ncias da Universidade do Porto && Folha \bf{{#2}}\\
Departamento de Matem�tica&& #1\\
Christian Lomp&&
\end{tabular}
\bigskip\begin{center}
\bf{\Large{Exerc�cios de �lgebra Computacional (M342)}}\\[3mm] 
\end{center}\bigskip}

\pagestyle{empty}

%\flushright{ }
 \addtolength\topmargin{-2cm}
 \setlength{\textwidth}{17cm}
\addtolength\textheight{3cm} 
%\setlength{\textheight}{26cm}
 \setlength{\evensidemargin}{-5mm}
 \setlength{\oddsidemargin}{-5mm}

\usepackage{hyperref}
\newcommand{\real}[1]{\underline{\mathbf{#1}}} 

\begin{document}
\writetitle{12.09.2011}{1}{1}


\nextsect

\sectitle{Introdu��o a linguagem C++}


Uma introdu��o a linguagem C++ encontra-se nas p�ginas
\url{http://www.inf.ufrgs.br/~johann/cpp2004/} e \url{http://www.cplusplus.com/doc/tutorial/}.

\begin{enumerate}
\exitem Escreva um programa em C++ com as seguintes caracter�sticas:
\begin{verbatim} 
Input: um n�mero positivo q, um n�mero n
Output: representa��o q-�ria do n�mero n
\end{verbatim}

\exitem Crie a seguinte classe:
\begin{verbatim} 
class Inteiros {

    vector<int> coeficientes;
    unsigned int base;

  public:
    Inteiros(unsigned int q, int n);
  
    void Print();
    unsigned int Valor(); 
}
\end{verbatim}
Onde \begin{itemize}
      \item o ``construtor'' $Inteiros(q,n)$ calcula os d�gitos do inteiro $n$ na base $q$ e guarda-os no vector $coeficientes$;
      \item $Print()$ imprime a representa��o $q$-�ria do n�mero guardado;
      \item $Valor()$ calcule o valor decimal do inteiro na representa��o $q$-�ria (s� se o inteiro � menor do que $4294967295$).
     \end{itemize}

\exitem Calcue as representa��es $6$-�ria de $123$, $8$-�ria de $5667$ $256$-�ria de $98121298$.
\exitem Calcue o valor decimal das representa��es $(10514321)_4$, $(6514326)_8$ e $(21012)_3$.
\exitem Escreva uma fun��o que soma dois n�meros inteiros na representa��o $q$-�ria.
\end{enumerate}




\end{document}