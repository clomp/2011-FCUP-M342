\include{algcomp_folha_header}
\usepackage{hyperref}
\newcommand{\real}[1]{\underline{\mathbf{#1}}} 

\begin{document}
\writetitle{12.09.2011}{1}{1}


\nextsect

\sectitle{Introdu��o a linguagem C++}


Uma introdu��o a linguagem C++ encontra-se nas p�ginas
\url{http://www.inf.ufrgs.br/~johann/cpp2004/} e \url{http://www.cplusplus.com/doc/tutorial/}.

\begin{enumerate}
\exitem Escreva um programa em C++ que  com as seguintes caracter�sticas:
\begin{verbatim} 
Input: um n�mero positivo q, um n�mero n
Output: representa��o q-�ria do n�mero n
\end{verbatim}

\exitem Crie a seguinte classe:
\begin{verbatim} 
class Inteiros {

    vector<int> coeficientes;
    unsigned int base;

  public:
    Inteiros(unsigned int q, int n);
  
    void Print();
    int Valor();
}
\end{verbatim}
Onde \begin{itemize}
      \item o ``construtor'' $Inteiros(q,n)$ calcula os d�gitos do inteiro $n$ na base $q$ e guarda-os no vector $coeficientes$;
      \item $Print()$ imprime a representa��o $q$-�ria do n�mero guardado;
      \item $Valor()$ calcule o valor decimal do inteiro na representa��o $q$-�ria.
     \end{itemize}

\exitem Calcue as representa��es $6$-�ria de $123$, $8$-�ria de $5667$ $256$-�ria de $98121298$.
\exitem Calcue o valor decimal das representa��es $(10514321)_4$, $(6514326)_8$ e $(21012)_3$.
\exitem Escreva uma fun��o que soma dois n�meros inteiros na representa��o $q$-�ria.
\end{enumerate}




\end{document}