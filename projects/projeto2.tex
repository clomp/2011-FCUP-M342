\documentclass{article}
\usepackage{a4wide}
%\usepackage[latin1]{inputenc}
\usepackage{amssymb}
\usepackage{algorithmic}
\usepackage{polynom}
\usepackage[english, portuguese]{babel}
\usepackage{hyperref}
\usepackage{amssymb}
\usepackage[utf8x]{inputenc}
\usepackage{euler}
\usepackage{listings}
\lstset{language=C++, basicstyle=\footnotesize}

\newcommand{\nlim}[1]{\displaystyle{\lim_{n\rightarrow \infty} \: {#1}}}
\newcommand{\ser}{\sum_{n=1}^\infty \:}
\newcommand{\sern}[1]{{\sum_{n={#1}}^\infty\:}}
\newcommand{\dsum}{\displaystyle\sum}
\newcommand{\ZZ}{\mathbb{Z}}
\newcommand{\RR}{\mathbb{R}}
\newcommand{\NN}{\mathbb{N}}
\newcommand{\QQ}{\mathbb{Q}}
\newcommand{\CC}{\mathbb{C}}
\newcommand{\HH}{\mathbb{H}}
\newcommand{\mdc}{\mathrm{mdc}}
\newcommand{\arctg}{\mathrm{arc\:tg}}
\newcommand{\sen}[1]{\mathrm{sen}(#1)}
\renewcommand{\deg}[1]{\mathrm{grau}(#1)}
\newcommand{\Aut}[1]{\mathrm{Aut}(#1)}
\newcommand{\sgn}[1]{\mathrm{sgn}(#1)}
\newcommand{\MatrixZ}[4]{\left(\begin{array}{rr} #1 & #2 \\ #3 & #4 \end{array}\right)}
%\newcommand{\cos}{\mathrm{cos}}
\newcommand{\senh}{\mathrm{sh}}
%\newcommand{\cosh}{\mathrm{cosh}}
\newcommand{\len}{\mathrm{len}}
\newcommand{\arctgh}{\mathrm{arc\:tgh}}


\renewcommand{\algorithmicrequire}{\textbf{Input:}}
\renewcommand{\algorithmicensure}{\textbf{Output:}}
\newcommand{\q}{\mathbf{q}}

\newcommand{\grau}[1]{\mathrm{grau}({#1})}
\newcommand{\cvector}[1]{\mathrm{vector}(#1)}
\newcommand{\lc}[1]{\mathrm{lc}(#1)}


%\newcounter{kapitel}
\newcounter{abschnitt}
\newcounter{uebung}
\newcommand{\reset}[1]{\setcounter{#1}{1}}
%\newcommand{\nextchap}{\addtocounter{kapitel}{1} \reset{abschnitt}\reset{uebung}}
\newcommand{\nextsect}{\addtocounter{abschnitt}{1} \reset{uebung} }
\newcommand{\nextexer}{\addtocounter{uebung}{1}}
%\newcommand{\getex}{(\arabic{kapitel}.\arabic{abschnitt}.\arabic{uebung})}
\newcommand{\getex}{(\arabic{abschnitt}.\arabic{uebung})}
\newcommand{\exitem}{\item[\getex] \nextexer}
\newcommand{\tpcexitem}{\item[\tpc \getex] \nextexer}
\renewcommand{\exp}[1]{\mathbf{e}^{#1}}
\newcommand{\sectitle}[1]{\begin{center}{\bf{\arabic{abschnitt}. \underline{#1}}}\end{center}}

%\setlength{\marginparwidth}{1.2in}
%\let\oldmarginpar\marginpar
%\renewcommand\marginpar[1]{\-\oldmarginpar[\raggedleft\footnotesize #1]%
%{\raggedright\footnotesize #1}}

\newcommand{\tpc}{{\bf $\mathbf{\star} \:\: $}}



%\newcommand{\writetitle}[6]{\bigskip \setcounter{kapitel}{#3}\setcounter{abschnitt}{#4}\reset{uebung}
\newcommand{\writetitle}[6]{\bigskip \setcounter{abschnitt}{#3}\reset{uebung}
\begin{tabular}{lp{6cm}r}
Faculdade de Ciências da Universidade do Porto && Projeto \bf{{#2}}\\
Departamento de Matemática&& #1\\
Christian Lomp&&
\end{tabular}
\bigskip\begin{center}
\bf{\Large{Projeto de Álgebra Computacional (M342)}}\\[3mm] 
\end{center}\bigskip}

\pagestyle{empty}

%\flushright{ }
 \addtolength\topmargin{-2cm}
 \setlength{\textwidth}{17cm}
\addtolength\textheight{3cm} 
%\setlength{\textheight}{26cm}
 \setlength{\evensidemargin}{-5mm}
 \setlength{\oddsidemargin}{-5mm}

\usepackage{hyperref}
\newcommand{\shift}[1]{\setlength{\leftskip}{15 mm}{#1}\medskip}

\newcommand{\poli}{{$< polinomio >$ }}

\begin{document}
\writetitle{entrega: 5.12.2011}{2}{2}

\nextsect

\sectitle{A classe dos polinómios}

{\bf Objetivo:} Definir uma classe em C++ que permita a aritmética de polinómios $\ZZ_2[x]$ numa indeterminada sobre o corpo $\ZZ_2$.

\nextexer{\bf Especificação da classe:} A classe \poli deve representar polinomios com coeficientes $\{0,1\}$
utilizando a classe $< vector >$.

A classe \poli deve ter a seguinte estrutura:

\begin{lstlisting}[language=C++]
class polinomio {

  bool nulo; 
  vector <bool> coeficientes;

  public:
    polinomio();
    polinomio(bool, vector<unsigned int>);
    bool PolinomioNulo();
    polinomio operator + (polinomio);
    polinomio operator - (polinomio);
    polinomio operator * (polinomio);
    polinomio operator / (polinomio);
    polinomio operator % (polinomio);	
    int Grau();	
    void Print();
}
\end{lstlisting}

\nextexer{\bf Especificações das funções:}

\begin{enumerate}
\item $nulo$ é $true$ se e só se o objeto da classe \poli representa o polinómio nulo.
\item $coeficientes$ é a lista dos coeficientes do polinómio $f=\sum_{i=0}^n a_i x^i \in \ZZ_2[x]$ que o objeto representa. O coeficiente $coeficientes[i]$ é $false$ se e só se $a_i=0$. (Nota que a linguagem $C$ representá o valor booleano $false$ por $0$ e $true$ por um inteiro não nulo.)
\item \setlength{\leftskip}{4mm} {\bf O construtor:}
\begin{lstlisting}[language=C++]
polinomio::polinomio() { ... }
\end{lstlisting}
\shift{Criação dum objecto da classe \poli que representa um elemento de $\ZZ_2[x]$.}
\begin{lstlisting}[language=C++]
polinomio::polinomio(bool nulo, vector<unsigned int> coef) { ... }
\end{lstlisting}
\shift{ Criação dum objecto da classe \poli  que representa o $f=\sum_{i=0}^n a_i x^i$ com $a_i=coef[i]$ ou $f=0$ se $nulo=true$.}

\item \setlength{\leftskip}{4mm} {\bf Aritmética:}
\begin{lstlisting}[language=C++]
polinomio polinomio::operator + (polinomio b) { ... }
\end{lstlisting}
\shift{O valor de retorno é um objeto da classe \poli  que representa a soma dos polinomios representados pelo objeto atual e pelo objeto $b$.}
\begin{lstlisting}[language=C++]
polinomio polinomio::operator - (polinomio b) { ... }
\end{lstlisting}
\shift{O valor de retorno é um objeto da classe \poli  que representa a diferenca dos polinomio representados pelo objeto atual e pelo objeto $b$.}
\begin{lstlisting}[language=C++]
polinomio polinomio::operator * (polinomio b) { ... }
\end{lstlisting}
\shift{O valor de retorno é um objeto da classe \poli  que representa a multiplicação dos polinomio representados pelo objeto atual e pelo objeto $b$.}
\begin{lstlisting}[language=C++]
polinomio polinomio::operator / (polinomio b) { ... }
\end{lstlisting}
\shift{O valor de retorno é um objeto da classe \poli que representa o quociente da divisão dos polinomios representados pelo objeto atual e pelo objeto $b$.}
\begin{lstlisting}[language=C++]
polinomio polinomio::operator % (polinomio b) { ... }
\end{lstlisting}
\shift{O valor de retorno é um objeto da classe \poli que representa o resto da divisão dos polinomio representados pelo objeto atual e pelo objeto $b$.}
\item \setlength{\leftskip}{4mm} {\bf Funções adicionais:}
\begin{lstlisting}[language=C++]
int polinomio::Grau();
\end{lstlisting}
\shift{Determine o grau do polinómio. Se o polinómio for nulo deve devolver $-1$.}

\begin{lstlisting}[language=C++]
void polinomio::Print();
\end{lstlisting}
\shift{Imprime o objeto da classe \poli .}
\end{enumerate}

\nextsect
\sectitle{O máximo divisor comum}

\nextexer{} Escreva uma função {\bf polinomio mdc(polinomio f, polinomio g)} que permite calcular o máximo divisor comum entre $f$ e $g$. Calcue $mdc(f,g)$ para $f=x^{11}+x^6+x^2+1$ e $g=x^6+x^3+1$.

\nextsect
\sectitle{Polinómios irredutíveis}

\nextexer{} Um polinómio $f\in \ZZ_2[x]\setminus \{0,1\}$ diz-se irredutível se para qualquer $g,h\in \ZZ_2[x]$ com $f=g*h$ temos que $f=1$ ou $g=1$, ou seja se os únicos divisores de $f$ são $1$ e $f$. Escreva uma função {\bf bool Irredutivel(polinomio f)} que permite determinar se um polinómio $f$ é irredutível ou não. Verifique quais dos polinómios são irredutíveis: 
\begin{enumerate}
\item $g=x^6+x^3+1$;
\item $h=x^6+x^3+x^2+x+1$;
\item $k=x^{10}+x^9+x^8+x^7+x^6+x^5+x^4+x^3+x^2+x+1$.
\end{enumerate}




\end{document}



