\include{project_header}
\usepackage{hyperref}
\newcommand{\shift}[1]{\setlength{\leftskip}{15 mm}{#1}\medskip}



\begin{document}
\writetitle{para entregar até 26.10.2011}{1}{1}

\nextsect

\sectitle{A classe inteiro}

{\bf Objetivo:} Definir uma classe em C++ que permite a aritmética de inteiros em precisão arbitrária (``arbitrary-precision arithmetic'').

\nextexer{\bf Especificação da classe:} A classe $<$inteiro $>$ deve representar inteiros na base $\q=10^9$ e deve 
utilizar a classe $<$vector$>$ para guardar os coeficientes da representação $\q$-ária.

A classe $<$inteiro$>$ deve ter a seguinte estrutura:

\begin{lstlisting}[language=C++]
class inteiro {

  bool negativo; 
  vector <unsigned int> coeficientes;

  public:
    inteiro();
    inteiro(int n);
    inteiro(bool, vector<unsigned int>);
    bool operator < (inteiro);
    bool operator > (inteiro);
    bool operator == (inteiro);
    bool operator != (inteiro);
    inteiro operator + (inteiro);
    inteiro operator - (inteiro);
    inteiro operator * (inteiro);
    inteiro operator / (inteiro);
    inteiro operator % (inteiro);

    string ConvertToString();
    inteiro ConvertToInteiro(string);

}
\end{lstlisting}

\nextexer{\bf Especificações das funções:}

\begin{enumerate}
\item \setlength{\leftskip}{4mm} {\bf O construtor:}
\begin{lstlisting}[language=C++]
inteiro::inteiro() { ... }
\end{lstlisting}
\shift{Criação dum objecto {\it inteiro} que representa o número $0$.}
\begin{lstlisting}[language=C++]
inteiro::inteiro(int n) { ... }
\end{lstlisting}
\shift{Criação dum objecto {\it inteiro} que representa o número $n$.}
\begin{lstlisting}[language=C++]
inteiro::inteiro(bool sinal, vector<unsigned int> coef) { ... }
\end{lstlisting}
\shift{ Criação dum objecto {\it inteiro} que representa o número dado pela representação $(sinal, coef)$.}

\item \setlength{\leftskip}{4mm} {\bf Comparações:}
\begin{lstlisting}[language=C++]
bool inteiro::operator == (inteiro b) { ... }
\end{lstlisting}
\shift{O valor de retorno é {\it true} se os inteiros representados pelo objeto actual e pelo objeto $b$ são iguais.}
\begin{lstlisting}[language=C++]
bool inteiro::operator != (inteiro b) { ... }
\end{lstlisting}
\shift{O valor de retorno é {\it true} se os inteiros representados pelo objeto actual e pelo objeto $b$ não são iguais.}
\begin{lstlisting}[language=C++]
bool inteiro::operator < (inteiro b) { ... }
\end{lstlisting}
\shift{O valor de retorno é {\it true} se o inteiros representado pelo objeto actual é menor do que o inteiro representado pelo objeto $b$.}
\begin{lstlisting}[language=C++]
bool inteiro::operator > (inteiro b) { ... }
\end{lstlisting}
\shift{O valor de retorno é {\it true} se o inteiros representado pelo objeto actual é maior do que o inteiro representado pelo objeto $b$.}

\item \setlength{\leftskip}{4mm} {\bf Aritmética:}
\begin{lstlisting}[language=C++]
inteiro inteiro::operator + (inteiro b) { ... }
\end{lstlisting}
\shift{O valor de retorno é um objeto {\it inteiro} que representa a soma dos inteiros representados pelo objeto actual e pelo objeto $b$.}
\begin{lstlisting}[language=C++]
inteiro inteiro::operator - (inteiro b) { ... }
\end{lstlisting}
\shift{O valor de retorno é um objeto {\it inteiro} que representa a diferenca dos inteiros representados pelo objeto actual e pelo objeto $b$.}
\begin{lstlisting}[language=C++]
inteiro inteiro::operator * (inteiro b) { ... }
\end{lstlisting}
\shift{O valor de retorno é um objeto {\it inteiro} que representa a multiplicação dos inteiros representados pelo objeto actual e pelo objeto $b$.}
\begin{lstlisting}[language=C++]
inteiro inteiro::operator / (inteiro b) { ... }
\end{lstlisting}
\shift{O valor de retorno é um objeto {\it inteiro} que representa o quociente da divisão dos inteiros representados pelo objeto actual e pelo objeto $b$.}
\begin{lstlisting}[language=C++]
inteiro inteiro::operator % (inteiro b) { ... }
\end{lstlisting}
\shift{O valor de retorno é um objeto {\it inteiro} que representa o resto da divisão dos inteiros representados pelo objeto actual e pelo objeto $b$.}
\end{enumerate}

\sectitle{Problemas adicionais}

\nextexer{} Escreva uma função {\bf inteiro factorial(inteiro)} que permite calcular o factorial de um inteiro. Calcue $100!$ e $1000!$.

\nextexer{} Escreva uma função {\bf inteiro dvisor(inteiro)} que encontre o menor divisor maior que $1$ de um inteiro. Factorize o número $10000000001$ em números primos.
\newpage


\sectitle{Recurso Auxiliar}

Podemo utilisar as seguintes funções que permite a conversação dos coeficientes entre o formato {\it vectores$<$unsigned int$>$} e o formato de texto {\it string}.

Para utilizar as funções precisa de incluir as bibliotecas $<$vector$>$, $<$string$>$, $<$cstdlib$>$ e $<$sstream$>$.

\begin{lstlisting}[language=C++]
#include <vector>
#include <string>
#include <cstdlib>
#include <sstream>
\end{lstlisting}

BASE e BASE\_LEN são constantes que são preciso para a conversação.
\begin{lstlisting}[language=C++]
#define BASE 1000000000
#define BASE_LEN 9


string ConvertVectorToString(vector<unsigned int> v)
{
	int n=v.size();
	ostringstream vector_string;
	for(int i=n-1;i>=0;i--)
	{
		if(i<n-1)
		{
		  int r=v[i]%BASE;
		  for(int j=0;j<BASE_LEN-r;j++)
			vector_string << "0";
		};
		vector_string << v[i];
        };
	return vector_string.str();
};

vector<unsigned int> ConvertStringToVector(string s)
{
	vector<unsigned int> v;
        int n=s.length();
	int q=n/BASE_LEN;
	int r=n%BASE_LEN;
	string sub;
        for(int i=q-1;i>=0;i--)
	{
		sub=s.substr(r+i*BASE_LEN,BASE_LEN);				
		v.push_back(atoi(sub.c_str()));
	};
	if(r>0)
	{ 
		sub=s.substr(0,r);
		v.push_back(atoi(sub.c_str()));
	};
        return v;
};
\end{lstlisting}




\end{document}



