\include{project_header}
\usepackage{hyperref}
\newcommand{\shift}[1]{\setlength{\leftskip}{15 mm}{#1}\medskip}



\begin{document}
\writetitle{12.09.2011}{1}{1}

\nextsect

\sectitle{A classe inteiro}

{\bf Objetivo:} Definir uma classe em C++ que permite a aritmética de inteiros em precisão arbitrária (``arbitrary-precision arithmetic'').

\nextexer{\bf Especificação da classe:} A classe $<$inteiro $>$ deve representar inteiros na base $\q=10^9$ e deve 
utilizar a classe $<$vector$>$ para guardar os coeficientes da representação $\q$-ária.

A classe $<$inteiro$>$ deve ter a seguinte estrutura:

\begin{lstlisting}[language=C++]
class inteiro {

  bool negativo; 
  vector <unsigned int> coeficientes;

  public:
    inteiro();
    inteiro(int n);
    inteiro(bool, vector<unsigned int>);
    void operator = (inteiro);
    bool operator < (inteiro);
    bool operator == (inteiro);
    inteiro operator + (inteiro);
    inteiro operator - (inteiro);
    inteiro operator * (inteiro);
    inteiro operator / (inteiro);
    inteiro operator % (inteiro);

    string ConvertToString();
    inteiro ConvertToInteiro(string);

}
\end{lstlisting}

\nextexer{\bf Especificações das funções:}

\begin{enumerate}
\item \setlength{\leftskip}{4mm} {\bf O construtor:}
\begin{lstlisting}[language=C++]
inteiro::inteiro() { ... }
\end{lstlisting}
\shift{Criação dum objecto {\it inteiro} que representa o número $0$.}
\begin{lstlisting}[language=C++]
inteiro::inteiro(int n) { ... }
\end{lstlisting}
\shift{Criação dum objecto {\it inteiro} que representa o número $n$.}
\begin{lstlisting}[language=C++]
inteiro::inteiro(bool sinal, vector<unsigned int> coef) { ... }
\end{lstlisting}
\shift{ Criação dum objecto {\it inteiro} que representa o número dado pela representação $(sinal, coef)$.}

\item \setlength{\leftskip}{4mm} {\bf Comparações:}
\begin{lstlisting}[language=C++]
bool inteiro::operator == (inteiro b) { ... }
\end{lstlisting}
\shift{O valor de retorno é {\it true} se os inteiros representados pelo objeto actual e pelo objeto $b$ são iguais.}
\begin{lstlisting}[language=C++]
bool inteiro::operator != (inteiro b) { ... }
\end{lstlisting}
\shift{O valor de retorno é {\it true} se os inteiros representados pelo objeto actual e pelo objeto $b$ não são iguais.}
\begin{lstlisting}[language=C++]
bool inteiro::operator < (inteiro b) { ... }
\end{lstlisting}
\shift{O valor de retorno é {\it true} se o inteiros representado pelo objeto actual é menor do que o inteiro representado pelo objeto $b$.}
\begin{lstlisting}[language=C++]
bool inteiro::operator > (inteiro b) { ... }
\end{lstlisting}
\shift{O valor de retorno é {\it true} se o inteiros representado pelo objeto actual é maior do que o inteiro representado pelo objeto $b$.}

\item \setlength{\leftskip}{4mm} {\bf Atribuir:}
\begin{lstlisting}[language=C++]
void inteiro::operator = (inteiro b) { ... }
\end{lstlisting}
\shift{A representação do inteiro $b$ será copiada para o objeto actual.}

\item \setlength{\leftskip}{4mm} {\bf Aritmética:}
\begin{lstlisting}[language=C++]
inteiro inteiro::operator + (inteiro b) { ... }
\end{lstlisting}
\shift{O valor de retorno é um objeto {\it inteiro} que representa a soma dos inteiros representados pelo objeto actual e pelo objeto $b$.}
\begin{lstlisting}[language=C++]
inteiro inteiro::operator - (inteiro b) { ... }
\end{lstlisting}
\shift{O valor de retorno é um objeto {\it inteiro} que representa a diferenca dos inteiros representados pelo objeto actual e pelo objeto $b$.}
\begin{lstlisting}[language=C++]
inteiro inteiro::operator * (inteiro b) { ... }
\end{lstlisting}
\shift{O valor de retorno é um objeto {\it inteiro} que representa a multiplicação dos inteiros representados pelo objeto actual e pelo objeto $b$.}
\begin{lstlisting}[language=C++]
inteiro inteiro::operator / (inteiro b) { ... }
\end{lstlisting}
\shift{O valor de retorno é um objeto {\it inteiro} que representa o quociente da divisão dos inteiros representados pelo objeto actual e pelo objeto $b$.}
\begin{lstlisting}[language=C++]
inteiro inteiro::operator % (inteiro b) { ... }
\end{lstlisting}
\shift{O valor de retorno é um objeto {\it inteiro} que representa o resto da divisão dos inteiros representados pelo objeto actual e pelo objeto $b$.}
\end{enumerate}


\end{document}



