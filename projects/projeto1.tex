\documentclass{article}
\usepackage{a4wide}
%\usepackage[latin1]{inputenc}
\usepackage{amssymb}
\usepackage{algorithmic}
\usepackage{polynom}
\usepackage[english, portuguese]{babel}
\usepackage{hyperref}
\usepackage{amssymb}
\usepackage[utf8x]{inputenc}
\usepackage{euler}
\usepackage{listings}
\lstset{language=C++, basicstyle=\footnotesize}

\newcommand{\nlim}[1]{\displaystyle{\lim_{n\rightarrow \infty} \: {#1}}}
\newcommand{\ser}{\sum_{n=1}^\infty \:}
\newcommand{\sern}[1]{{\sum_{n={#1}}^\infty\:}}
\newcommand{\dsum}{\displaystyle\sum}
\newcommand{\ZZ}{\mathbb{Z}}
\newcommand{\RR}{\mathbb{R}}
\newcommand{\NN}{\mathbb{N}}
\newcommand{\QQ}{\mathbb{Q}}
\newcommand{\CC}{\mathbb{C}}
\newcommand{\HH}{\mathbb{H}}
\newcommand{\mdc}{\mathrm{mdc}}
\newcommand{\arctg}{\mathrm{arc\:tg}}
\newcommand{\sen}[1]{\mathrm{sen}(#1)}
\renewcommand{\deg}[1]{\mathrm{grau}(#1)}
\newcommand{\Aut}[1]{\mathrm{Aut}(#1)}
\newcommand{\sgn}[1]{\mathrm{sgn}(#1)}
\newcommand{\MatrixZ}[4]{\left(\begin{array}{rr} #1 & #2 \\ #3 & #4 \end{array}\right)}
%\newcommand{\cos}{\mathrm{cos}}
\newcommand{\senh}{\mathrm{sh}}
%\newcommand{\cosh}{\mathrm{cosh}}
\newcommand{\len}{\mathrm{len}}
\newcommand{\arctgh}{\mathrm{arc\:tgh}}


\renewcommand{\algorithmicrequire}{\textbf{Input:}}
\renewcommand{\algorithmicensure}{\textbf{Output:}}
\newcommand{\q}{\mathbf{q}}

\newcommand{\grau}[1]{\mathrm{grau}({#1})}
\newcommand{\cvector}[1]{\mathrm{vector}(#1)}
\newcommand{\lc}[1]{\mathrm{lc}(#1)}


%\newcounter{kapitel}
\newcounter{abschnitt}
\newcounter{uebung}
\newcommand{\reset}[1]{\setcounter{#1}{1}}
%\newcommand{\nextchap}{\addtocounter{kapitel}{1} \reset{abschnitt}\reset{uebung}}
\newcommand{\nextsect}{\addtocounter{abschnitt}{1} \reset{uebung} }
\newcommand{\nextexer}{\addtocounter{uebung}{1}}
%\newcommand{\getex}{(\arabic{kapitel}.\arabic{abschnitt}.\arabic{uebung})}
\newcommand{\getex}{(\arabic{abschnitt}.\arabic{uebung})}
\newcommand{\exitem}{\item[\getex] \nextexer}
\newcommand{\tpcexitem}{\item[\tpc \getex] \nextexer}
\renewcommand{\exp}[1]{\mathbf{e}^{#1}}
\newcommand{\sectitle}[1]{\begin{center}{\bf{\arabic{abschnitt}. \underline{#1}}}\end{center}}

%\setlength{\marginparwidth}{1.2in}
%\let\oldmarginpar\marginpar
%\renewcommand\marginpar[1]{\-\oldmarginpar[\raggedleft\footnotesize #1]%
%{\raggedright\footnotesize #1}}

\newcommand{\tpc}{{\bf $\mathbf{\star} \:\: $}}



%\newcommand{\writetitle}[6]{\bigskip \setcounter{kapitel}{#3}\setcounter{abschnitt}{#4}\reset{uebung}
\newcommand{\writetitle}[6]{\bigskip \setcounter{abschnitt}{#3}\reset{uebung}
\begin{tabular}{lp{6cm}r}
Faculdade de Ciências da Universidade do Porto && Projeto \bf{{#2}}\\
Departamento de Matemática&& #1\\
Christian Lomp&&
\end{tabular}
\bigskip\begin{center}
\bf{\Large{Projeto de Álgebra Computacional (M342)}}\\[3mm] 
\end{center}\bigskip}

\pagestyle{empty}

%\flushright{ }
 \addtolength\topmargin{-2cm}
 \setlength{\textwidth}{17cm}
\addtolength\textheight{3cm} 
%\setlength{\textheight}{26cm}
 \setlength{\evensidemargin}{-5mm}
 \setlength{\oddsidemargin}{-5mm}

\usepackage{hyperref}
\newcommand{\shift}[1]{\setlength{\leftskip}{15 mm}{#1}\medskip}



\begin{document}
\writetitle{12.09.2011}{1}{1}

\nextsect

\sectitle{A classe inteiro}

{\bf Objetivo:} Definir uma classe em C++ que permite a aritmética de inteiros em precisão arbitrária (``arbitrary-precision arithmetic'').

\nextexer{\bf Especificação da classe:} A classe $<$inteiro $>$ deve representar inteiros na base $\q=10^9$ e deve 
utilizar a classe $<$vector$>$ para guardar os coeficientes da representação $\q$-ária.

A classe $<$inteiro$>$ deve ter a seguinte estrutura:

\begin{lstlisting}[language=C++]
class inteiro {

  bool negativo; 
  vector <unsigned int> coeficientes;

  public:
    inteiro();
    inteiro(int n);
    inteiro(bool, vector<unsigned int>);
    void operator = (inteiro);
    bool operator < (inteiro);
    bool operator == (inteiro);
    inteiro operator + (inteiro);
    inteiro operator - (inteiro);
    inteiro operator * (inteiro);
    inteiro operator / (inteiro);
    inteiro operator % (inteiro);

    string ConvertToString();
    inteiro ConvertToInteiro(string);

}
\end{lstlisting}

\nextexer{\bf Especificações das funções:}

\begin{enumerate}
\item \setlength{\leftskip}{4mm} {\bf O construtor:}
\begin{lstlisting}[language=C++]
inteiro::inteiro() { ... }
\end{lstlisting}
\shift{Criação dum objecto {\it inteiro} que representa o número $0$.}
\begin{lstlisting}[language=C++]
inteiro::inteiro(int n) { ... }
\end{lstlisting}
\shift{Criação dum objecto {\it inteiro} que representa o número $n$.}
\begin{lstlisting}[language=C++]
inteiro::inteiro(bool sinal, vector<unsigned int> coef) { ... }
\end{lstlisting}
\shift{ Criação dum objecto {\it inteiro} que representa o número dado pela representação $(sinal, coef)$.}

\item \setlength{\leftskip}{4mm} {\bf Comparações:}
\begin{lstlisting}[language=C++]
bool inteiro::operator == (inteiro b) { ... }
\end{lstlisting}
\shift{O valor de retorno é {\it true} se os inteiros representados pelo objeto actual e pelo objeto $b$ são iguais.}
\begin{lstlisting}[language=C++]
bool inteiro::operator != (inteiro b) { ... }
\end{lstlisting}
\shift{O valor de retorno é {\it true} se os inteiros representados pelo objeto actual e pelo objeto $b$ não são iguais.}
\begin{lstlisting}[language=C++]
bool inteiro::operator < (inteiro b) { ... }
\end{lstlisting}
\shift{O valor de retorno é {\it true} se o inteiros representado pelo objeto actual é menor do que o inteiro representado pelo objeto $b$.}
\begin{lstlisting}[language=C++]
bool inteiro::operator > (inteiro b) { ... }
\end{lstlisting}
\shift{O valor de retorno é {\it true} se o inteiros representado pelo objeto actual é maior do que o inteiro representado pelo objeto $b$.}

\item \setlength{\leftskip}{4mm} {\bf Atribuir:}
\begin{lstlisting}[language=C++]
void inteiro::operator = (inteiro b) { ... }
\end{lstlisting}
\shift{A representação do inteiro $b$ será copiada para o objeto actual.}

\item \setlength{\leftskip}{4mm} {\bf Aritmética:}
\begin{lstlisting}[language=C++]
inteiro inteiro::operator + (inteiro b) { ... }
\end{lstlisting}
\shift{O valor de retorno é um objeto {\it inteiro} que representa a soma dos inteiros representados pelo objeto actual e pelo objeto $b$.}
\begin{lstlisting}[language=C++]
inteiro inteiro::operator - (inteiro b) { ... }
\end{lstlisting}
\shift{O valor de retorno é um objeto {\it inteiro} que representa a diferenca dos inteiros representados pelo objeto actual e pelo objeto $b$.}
\begin{lstlisting}[language=C++]
inteiro inteiro::operator * (inteiro b) { ... }
\end{lstlisting}
\shift{O valor de retorno é um objeto {\it inteiro} que representa a multiplicação dos inteiros representados pelo objeto actual e pelo objeto $b$.}
\begin{lstlisting}[language=C++]
inteiro inteiro::operator / (inteiro b) { ... }
\end{lstlisting}
\shift{O valor de retorno é um objeto {\it inteiro} que representa o quociente da divisão dos inteiros representados pelo objeto actual e pelo objeto $b$.}
\begin{lstlisting}[language=C++]
inteiro inteiro::operator % (inteiro b) { ... }
\end{lstlisting}
\shift{O valor de retorno é um objeto {\it inteiro} que representa o resto da divisão dos inteiros representados pelo objeto actual e pelo objeto $b$.}
\end{enumerate}


\end{document}



