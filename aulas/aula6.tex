%\documentclass[handout]{beamer}
\documentclass{beamer}
\usepackage{algorithmic}
\usepackage{polynom}
\usepackage[english, portuguese]{babel}
\usepackage{hyperref}
\usepackage{amssymb}
\usepackage[utf8x]{inputenc}
\usepackage{euler}

\usepackage[center]{caption}

\usepackage{listings}
\lstset{language=C++, basicstyle=\footnotesize}

%\floatname{algorithm}{Função }
\renewcommand{\algorithmicrequire}{\textbf{Input:}}
\renewcommand{\algorithmicensure}{\textbf{Output:}}
\newcommand{\q}{\mathbf{q}}
\newcommand{\ZZ}{\mathbb Z}
\newcommand{\NN}{\mathbb N}
\newcommand{\CC}{\mathbb C}
\newcommand{\RR}{\mathbb R}
\newcommand{\QQ}{\mathbb Q}
\newcommand{\grau}[1]{\mathrm{grau}({#1})}
\newcommand{\mdc}[2]{\mathrm{mdc}({#1}, {#2})}
\newcommand{\mmc}[2]{\mathrm{mmc}({#1}, {#2})}
\newcommand{\quo}[2]{{#1}\, \mathrm{quo} \, {#2}}
\newcommand{\rem}[2]{{#1}\, \mathrm{rem} \, {#2}}

\newcommand{\cvector}[1]{\mathrm{vector}(#1)}
\newcommand{\lc}[1]{\mathrm{lc}(#1)}
%\usetheme{Berlin}
\usetheme[pageofpages=of,% String used between the current page and the
                         % total page count.
          bullet=circle,% Use circles instead of squares for bullets.
          titleline=true,% Show a line below the frame title.
          alternativetitlepage=true,% Use the fancy title page.
          %titlepagelogo=logotipo_pdf1,% Logo for the first page.
          %watermark=logotipo_pdf1,% Watermark used in every page.
          %watermarkheight=100px,% Height of the watermark.
          %watermarkheightmult=2,% The watermark image is 4 times bigger
                                % than watermarkheight.
          ]{Torino}

\title[M342]{M342 Álgebra Computacional}
\author{Christian Lomp}
\institute{FCUP}
\date{10 de Outubro de 2011}
\begin{document}

\begin{frame}
\titlepage
\end{frame}






\begin{frame}{\bf Equações Diofantinas Lineares}
A equação diofantinas lineares:
$$ a_1x_1 + a_2x_2 + \ldots + a_n x_n = c,$$
com $a_1, \ldots, a_n, c \in \ZZ$ tem soluções $(x_1, \ldots, x_n)\in \ZZ^n$ se e só se $\mathrm{mdc}(a_1,\ldots, a_n) | c$.
\end{frame}

\begin{frame}{\bf Equações Diofantinas Lineares}
$$ax+by=c\qquad \Leftrightarrow \qquad a'x+b'y = c'$$
onde $a'=\frac{a}{\mdc{a}{b}}, b'=\frac{b}{\mdc{a}{b}}, c'=\frac{c}{\mdc{a}{b}}$.

Existem $s,t\in \ZZ$ tais que $1=sa'+tb'$ e $c'=c'sa'+c'tb'$.

\begin{eqnarray*}
&\Rightarrow &a'(x-c's)+b'(y-c't) = c'-c'=0\\
&\Leftrightarrow& a'(x-c's)=b'(c't-y) \\
&\Rightarrow& a' | (c't-y) \qquad \mbox{ porque } \mdc{a'}{b'}=1\\
&\Rightarrow& c't-y = a'n, \qquad n\in \ZZ \\
&\Rightarrow& y = c't-a'n, \qquad n\in \ZZ \\
&\Rightarrow& x = c't+b'n, \qquad n\in \ZZ
\end{eqnarray*}
\end{frame}

\begin{frame}[fragile]{A classe dos inteiros módulares}
\lstset{language=C++}
\begin{lstlisting}

class modular
{
	unsigned int modulo;
	int numero;
	
	int mdc(int,int);
	
	public: 
		modular(unsigned int,int);
		modular operator + (modular);
		modular operator - (modular);
		modular operator * (modular);
		bool invertivel();
		modulear inverso ();
}
\end{lstlisting}
\end{frame}

\begin{frame}[fragile]{A classe dos inteiros módulares}
\lstset{language=C++,basicstyle=\tiny}
\begin{lstlisting}
modular(unsigned int n, int a)
{
	base=n;
	numero=a%n;
};

modular modular::operator + (modular b)
{
	if (b.base == base) 
		return modular(base,b.numero+numero)
};

bool invertivel()
{
	return (mdc(numero,base)==1);
};

\end{lstlisting}
\end{frame}

\begin{frame}[fragile]{Algoritmo de Euclid em C++}
\lstset{language=C++,basicstyle=\tiny}
\begin{lstlisting}
int mdc(int a, int b, int* x, int* y)
{
	int r0,s0,t0,r1,s1,t1;
	r0=a;	s0=1;	t0=0;
	r1=b;	s1=0;	t1=1;
	
	while(r1!=0)
	{
	  int q = r0/r1;	  
	  int h = r0%r1; r0=r1; r1=h;
	  h=s0-q*s1; s0=s1; s1=h;
	  h=t0-q*t1; t0=t1; t1=h;
	};
	(*x)=s0;
	(*y)=t0;
	return r0;
};

int main()
{
		int a,b,x,y;
		x=0;
		y=0;
		cin >> a >> b;
		cout << "mdc=" << mdc(a,b,&x,&y);
		cout << " = " << x << "*" << a << " + "<<  y << "*" << b << endl;
}
\end{lstlisting}
\end{frame}


\begin{frame}{\bf 2.5 O Algoritmo de Karatsuba}
Multiplicação rápida de polinómios.
\begin{figure}[h]
  \begin{center}
    \includegraphics[scale=2]{Anatolii_Karatsuba.jpg}
  \end{center}

  \caption*{Anatolii Karatsuba (1937-2008)}
  \label{demTP}
\end{figure}
\end{frame}

\begin{frame}{\bf 2.5 O Algoritmo de Karatsuba}
Multiplicações são "caras". O algoritmo classico da mulitplicação de dois polinómios de grau $n$ precisa $O(n^2)$ multiplicações:
$$ \left(\sum_{i=0}^n a_i x^i \right)\left( \sum_{j=0}^n b_j x^j \right) = \sum_{i=0}^n \sum_{j=0}^n a_ib_j x^{i+j}$$

Podemos fazer melhor ?
\end{frame}

\begin{frame}{\bf 2.5 O Algoritmo de Karatsuba}
Se $f=f_1 x^m + f_0$ e $g=g_1 x^m + g_0$ com $f_0,f_1,g_0,g_1$ polinómios de grau menor que $m$.
\begin{eqnarray*} 
fg &=& f_1g_1 x^{2m} + (f_1g_0+f_0g_1) x^m + f_0g_0 \qquad \mbox{\color{red}{ 4 multiplicações}} \\\pause
   &=& \underbrace{f_1g_1}_u x^{2m} + ( \underbrace{(f_0+f_1)(g_0+g_1)}_w-\underbrace{f_1g_1}_u-\underbrace{f_0g_0}_v) x^m + \underbrace{f_0g_0}_v \\\pause
   &=& u x^{2m} + ( w-u-v) x^m + v \qquad \mbox{ \color{green}{\bf 3 multiplicações}} 
\end{eqnarray*}
onde $$ u=f_1g_1 \qquad v=f_0g_0 \qquad w=(f_0+f_1)(g_0+g_1)$$\pause
pois 
$$ w=(f_0+f_1)(g_0+g_1)=f_0g_0+f_1g_0+f_0g_1+f_1g_1 = u+v + (f_0g_1+f_1g_0).$$
\end{frame}

\begin{frame}[fragile]{\bf 2.5 O Algoritmo de Karatsuba}{\bf Dividir e conquistar}
\begin{algorithmic}
\REQUIRE polinómios $f,g$
\ENSURE $Karatsuba(f,g)=fg$
	\STATE $m\leftarrow \lceil max(grau(f),grau(g))/2 \rceil$
    \IF {$m<2$} \RETURN {$f\cdot g$ multiplicação usual}
	\ELSE
		\STATE $u\leftarrow Karatsuba(f_1,g_1)$
		\STATE $v\leftarrow Karatsuba(f_0,g_0)$
		\STATE $w\leftarrow Karatsuba(f_0+f_1,g_0+g_1)$
		\RETURN {$u\cdot x^{2m} + (w-u-v)\cdot x^m + v$}
	\ENDIF
\end{algorithmic}
\end{frame}

\begin{frame}[fragile]{\bf 2.5 O Algoritmo de Karatsuba}{\bf Dividir e conquistar}
\begin{block}{Teorema}
O algoritmo de Karatsuba precisa $O(n^{\log_2(3)})$ multiplicações.
\end{block}\pause

\medskip

\begin{tabular}{|c|c|c|c|}\hline
$n=2^k$ & Clássico $4^k$& Karatsuba $3^k$ & percentagem \\\hline
$4$ & $16$ & $9$ & $\sim 56\%$\\\hline
$8$ & $64$ & $27$ & $\sim 42\%$ \\\hline
$16$ & $256$ & $81$ & $\sim 31\%$\\\hline
$32$ & $1024$ & $243$ & $\sim 23\%$\\\hline
$64$ & $4096$ & $729$ & $\sim 17\%$\\\hline
$128$ & $16384$ & $2187$ & $\sim 13\%$ \\\hline
$256$ & $65536$ & $6561$ & $\sim 10\%$\\\hline
\end{tabular}
\end{frame}

\begin{frame}{\bf Implementação do algoritmo de Karatsuba}
Decomposição $f=f_1 x^m + f_0$ pelo quociente e resto:
$$ f_1 = f / x^m \qquad f_0 = f \% x^m$$
que só ``cortam'' o vector dos coeficientes em duas partes, i.e. se 
$f=(a_0, a_1, \ldots, a_ n)$ então 
$$f_0 = f\%x^m = (a_0, \ldots, a_{m-1})$$
$$f_1 = f/x^m = (a_m, \ldots, a_n).$$

A multiplicação por $x^k$ é um ``shift'', i.e. 
$$f\cdot x^k = (\underbrace{0, \ldots, 0}_{k-\mbox{vezes}}, a_0, \ldots, a_n).$$
\end{frame}
\begin{frame}[fragile]
\lstset{language=C++}%,basicstyle=\tiny}
\begin{lstlisting}
polinomio polinomio::karatsuba(polinomio f, polinomio g)
{
  int maximo=max(f.coef.size(),g.coef.size());
  int d=maximo/2 + maximo%2;
  
  if (d<2) return f*g;
  else {
    polinomio f0=f.rem(d);    polinomio f1=f.quo(d);
    polinomio g0=g.rem(d);    polinomio g1=g.quo(d);

    polinomio u=karatsuba(f1,g1);
    polinomio v=karatsuba(f0,g0);
    polinomio w=karatsuba(f0+f1,g0+g1);

    return u.shift(2*d)+(w-u-v).shift(d)+v;
  };
};
\end{lstlisting}
\end{frame}

\begin{frame}[fragile]{\bf 2.5 O Algoritmo de Karatsuba}{\bf para inteiros}
\begin{algorithmic}
\REQUIRE inteiros $a=(a_0,\ldots, a_{n}),b=(b_0, \ldots, b_{k})$ na base $\q$
\ENSURE $Karatsuba(a,b)=ab$
	\STATE $m\leftarrow \lceil max(n,k)/2 \rceil$
    \IF {$m<2$} \RETURN {$a\cdot b$ multiplicação usual}
	\ELSE
		\STATE $a'\leftarrow (a_0,\ldots, a_{m-1}),\qquad a''\leftarrow (a_m,\ldots, a_{n});$
		\STATE $b'\leftarrow (b_0,\ldots, b_{m-1}),\qquad b''\leftarrow (b_m,\ldots, b_{k});$
		\STATE $u\leftarrow Karatsuba(a',b')$
		\STATE $v\leftarrow Karatsuba(a'',b'')$
		\STATE $w\leftarrow Karatsuba(a'+a'', b'+b'')$
		\RETURN {$v\cdot \q^{2m} + (w-u-v)\cdot \q^m + u$}
	\ENDIF
\end{algorithmic}
\end{frame}


\end{document}





2.2 A classes dos números racionais
2.3 A classes dos inteiros modulares
2.4 O template dos polinómios.
2.5 Multiplicação de polinómios: Algoritmo de Karatsuba

3. 
3.1 O template de vectores
3.2 O template de matrizes
3.3 Multiplicação rápida de matrizes: Algoritmo de Strassen.
3.4 Resolução de sistemas lineares

4. 
4.1 Introdução à geometria algébrica
4.2 Ordem nos monómios
4.3 Algoritmo da divisão para polinómios em múltiplos indeterminadas
4.4 Lemma de Dickson
4.5 Bases de Gröbner e Teorema de Hilbert
4.6 Algoritmo de Buchberger
4.7 Teoria de Eliminação
4.8 Teorema "Nullstellensatz" de Hilbert
4.9 Algoritmo de filiação de radicais
4.10 Aplicação à robótica
\end{frame}

