\documentclass[handout]{beamer}
\usepackage{algorithmic}
\usepackage{polynom}
\usepackage[english, portuguese]{babel}
\usepackage{hyperref}
\usepackage{amssymb}
\usepackage[utf8x]{inputenc}
\usepackage{euler}
\usepackage{listings}
\lstset{language=C++, basicstyle=\footnotesize}

%\floatname{algorithm}{Função }
\renewcommand{\algorithmicrequire}{\textbf{Input:}}
\renewcommand{\algorithmicensure}{\textbf{Output:}}
\newcommand{\q}{\mathbf{q}}
\newcommand{\ZZ}{\mathbb Z}
\newcommand{\NN}{\mathbb N}
\newcommand{\CC}{\mathbb C}
\newcommand{\RR}{\mathbb R}
\newcommand{\QQ}{\mathbb Q}
\newcommand{\grau}[1]{\mathrm{grau}({#1})}
\newcommand{\mdc}[2]{\mathrm{mdc}({#1}, {#2})}
\newcommand{\mmc}[2]{\mathrm{mmc}({#1}, {#2})}
\newcommand{\quo}[2]{{#1}\, \mathrm{quo} \, {#2}}
\newcommand{\rem}[2]{{#1}\, \mathrm{rem} \, {#2}}

\newcommand{\cvector}[1]{\mathrm{vector}(#1)}
\newcommand{\lc}[1]{\mathrm{lc}(#1)}
%\usetheme{Berlin}
\usetheme[pageofpages=of,% String used between the current page and the
                         % total page count.
          bullet=circle,% Use circles instead of squares for bullets.
          titleline=true,% Show a line below the frame title.
          alternativetitlepage=true,% Use the fancy title page.
          %titlepagelogo=logotipo_pdf1,% Logo for the first page.
          %watermark=logotipo_pdf1,% Watermark used in every page.
          %watermarkheight=100px,% Height of the watermark.
          %watermarkheightmult=2,% The watermark image is 4 times bigger
                                % than watermarkheight.
          ]{Torino}

\title[M342]{M342 Álgebra Computacional}
\author{Christian Lomp}
\institute{FCUP}
\date{26 de setembro de 2011}
\begin{document}

\begin{frame}
\titlepage
\end{frame}






\begin{frame}{\bf 2.2 Os numeros racionais} 

Seja $R$ um domínio integral e $Q$ o corpo das fracções, i.e.
$$Q= \left\{ \: \frac{a}{b} \: \mid a,b\in R \wedge b\neq 0 \right\}$$
onde $\frac{a}{b} := \{ (x,y) \in R^2 \mid y\neq 0 \wedge ay=bx.\}$.
\pause

\begin{enumerate}
\item $R=\ZZ$, $Q=\QQ$;
\item $R=\CC[x]$, $Q=\CC(x) = \mbox{ funções racionais}$;
\end{enumerate}
\end{frame}

\begin{frame}{\bf 2.2 Os numeros racionais} 

Seja $R$ um domínio Euclideano então 
$$ \frac{a}{b} = \frac{c}{d}$$
onde $a=c\mdc{a}{b}$ e $b=d=\mdc{a}{b}.$

Vamos supor que $\frac{a}{b}$ é sempre na forma reduzida, i.e. $\mdc{a}{b}=1$.
\pause

$$\frac{a}{b} + \frac{x}{y} = \frac{ay+bx}{xy}$$
$$\frac{a}{b} \cdot \frac{x}{y} = \frac{ax}{by}$$


\end{frame}

\begin{frame}[fragile]{A classe dos números racionais}
\lstset{language=C++}
\begin{lstlisting}
#include "inteiro.h"

class racional
{
	inteiro numerador;
	inteiro denumerador;
	
	public: 
		racional();
		racional(inteiro,inteiro);
		racional operator + (racional);
		racional operator * (racional);
		racional inverso();
		...
	
}
\end{lstlisting}
\end{frame}



\begin{frame}[fragile]{A classe dos números racionais}
\lstset{language=C++}
\begin{lstlisting}

racional racional::racional(inteiro n, inteiro d)
{
	numerador=n/mdc(n,d);
	denumerador=d/mdc(n,d);
};

racional racional::operator + (racional b)
{
	inteiro num=numerdor*b.denumerador + denumerador*b.numerador;
	inteiro denum=denumerador*b.denumerador;
	racional output(num,denum);
	return output;
};


\end{lstlisting}
\end{frame}

\begin{frame}{\bf O teorema chinês do resto}

Solução de sistemas de congruências de números relativamente primos $p_1, \ldots, p_k$
$$\left\{ \begin{array}{rcl} x &\equiv& a_1 \mod p_1 \\ x &\equiv& a_2 \mod p_2 \\ \vdots &&\\
x &\equiv& a_k \mod p_k\end{array}\right.$$
Seja $Q_j = \prod_{i\neq j} p_i$ para todo $1\leq j\leq k$. Então 
$$1=\mdc{Q_j}{p_j} = u_jQ_j + v_jp_j \qquad \Rightarrow \qquad  a_j \equiv a_ju_j Q_j \mod p_j.$$
A solução é portanto
$$ x = a_1u_1Q_1 + a_2u_2Q_2 + \ldots + a_ku_kQ_k \mod p_1\cdots p_k.$$


\end{frame}



\end{document}





2.2 A classes dos números racionais
2.3 A classes dos inteiros modulares
2.4 O template dos polinómios.
2.5 Multiplicação de polinómios: Algoritmo de Karatsuba

3. 
3.1 O template de vectores
3.2 O template de matrizes
3.3 Multiplicação rápida de matrizes: Algoritmo de Strassen.
3.4 Resolução de sistemas lineares

4. 
4.1 Introdução à geometria algébrica
4.2 Ordem nos monómios
4.3 Algoritmo da divisão para polinómios em múltiplos indeterminadas
4.4 Lemma de Dickson
4.5 Bases de Gröbner e Teorema de Hilbert
4.6 Algoritmo de Buchberger
4.7 Teoria de Eliminação
4.8 Teorema "Nullstellensatz" de Hilbert
4.9 Algoritmo de filiação de radicais
4.10 Aplicação à robótica
\end{frame}

