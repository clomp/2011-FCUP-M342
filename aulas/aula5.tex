%\documentclass[handout]{beamer}
\documentclass{beamer}
\usepackage{algorithmic}
\usepackage{polynom}
\usepackage[english, portuguese]{babel}
\usepackage{hyperref}
\usepackage{amssymb}
\usepackage[utf8x]{inputenc}
\usepackage{euler}
\usepackage{listings}
\lstset{language=C++, basicstyle=\footnotesize}

%\floatname{algorithm}{Função }
\renewcommand{\algorithmicrequire}{\textbf{Input:}}
\renewcommand{\algorithmicensure}{\textbf{Output:}}
\newcommand{\q}{\mathbf{q}}
\newcommand{\ZZ}{\mathbb Z}
\newcommand{\NN}{\mathbb N}
\newcommand{\CC}{\mathbb C}
\newcommand{\RR}{\mathbb R}
\newcommand{\QQ}{\mathbb Q}
\newcommand{\grau}[1]{\mathrm{grau}({#1})}
\newcommand{\mdc}[2]{\mathrm{mdc}({#1}, {#2})}
\newcommand{\mmc}[2]{\mathrm{mmc}({#1}, {#2})}
\newcommand{\quo}[2]{{#1}\, \mathrm{quo} \, {#2}}
\newcommand{\rem}[2]{{#1}\, \mathrm{rem} \, {#2}}

\newcommand{\cvector}[1]{\mathrm{vector}(#1)}
\newcommand{\lc}[1]{\mathrm{lc}(#1)}
%\usetheme{Berlin}
\usetheme[pageofpages=of,% String used between the current page and the
                         % total page count.
          bullet=circle,% Use circles instead of squares for bullets.
          titleline=true,% Show a line below the frame title.
          alternativetitlepage=true,% Use the fancy title page.
          %titlepagelogo=logotipo_pdf1,% Logo for the first page.
          %watermark=logotipo_pdf1,% Watermark used in every page.
          %watermarkheight=100px,% Height of the watermark.
          %watermarkheightmult=2,% The watermark image is 4 times bigger
                                % than watermarkheight.
          ]{Torino}

\title[M342]{M342 Álgebra Computacional}
\author{Christian Lomp}
\institute{FCUP}
\date{26 e 28 de setembro de 2011}
\begin{document}

\begin{frame}
\titlepage
\end{frame}






\begin{frame}{\bf 2.2 Os numeros racionais} 

Seja $R$ um domínio integral e $Q$ o corpo das fracções, i.e.
$$Q= \left\{ \: \frac{a}{b} \: \mid a,b\in R \wedge b\neq 0 \right\}$$
onde $\frac{a}{b} := \{ (x,y) \in R^2 \mid y\neq 0 \wedge ay=bx.\}$.
\pause

\begin{enumerate}
\item $R=\ZZ$, $Q=\QQ$;
\item $R=\CC[x]$, $Q=\CC(x) = \mbox{ funções racionais}$;
\end{enumerate}
\end{frame}

\begin{frame}{\bf 2.2 Os numeros racionais} 

Seja $R$ um domínio Euclideano então 
$$ \frac{a}{b} = \frac{c}{d}$$
onde $a=c\mdc{a}{b}$ e $b=d=\mdc{a}{b}.$

Vamos supor que $\frac{a}{b}$ é sempre na forma reduzida, i.e. $\mdc{a}{b}=1$.
\pause

$$\frac{a}{b} + \frac{x}{y} = \frac{ay+bx}{xy}$$
$$\frac{a}{b} \cdot \frac{x}{y} = \frac{ax}{by}$$


\end{frame}

\begin{frame}[fragile]{A classe dos números racionais}
\lstset{language=C++}
\begin{lstlisting}
#include "inteiro.h"

class racional
{
	inteiro numerador;
	inteiro denumerador;
	
	public: 
		racional();
		racional(inteiro,inteiro);
		racional operator + (racional);
		racional operator * (racional);
		racional inverso();
		...
	
}
\end{lstlisting}
\end{frame}



\begin{frame}[fragile]{A classe dos números racionais}
\lstset{language=C++}
\begin{lstlisting}

racional racional::racional(inteiro n, inteiro d)
{
	numerador=n/mdc(n,d);
	denumerador=d/mdc(n,d);
};

racional racional::operator + (racional b)
{
	inteiro num=numerdor*b.denumerador + denumerador*b.numerador;
	inteiro denum=denumerador*b.denumerador;
	racional output(num,denum);
	return output;
};


\end{lstlisting}
\end{frame}

\begin{frame}{\bf O teorema chinês do resto}

Solução de sistemas de congruências de números relativamente primos $p_1, \ldots, p_k$
$$\left\{ \begin{array}{rcl} x &\equiv& a_1 \mod p_1 \\ x &\equiv& a_2 \mod p_2 \\ \vdots &&\\
x &\equiv& a_k \mod p_k\end{array}\right.$$
Seja $Q_j = \prod_{i\neq j} p_i$ para todo $1\leq j\leq k$. Então 
$$1=\mdc{Q_j}{p_j} = u_jQ_j + v_jp_j \qquad \Rightarrow \qquad  a_j \equiv a_ju_j Q_j \mod p_j.$$
A solução é portanto
$$ x = a_1u_1Q_1 + a_2u_2Q_2 + \ldots + a_ku_kQ_k \mod p_1\cdots p_k.$$


\end{frame}

%%%%%%%%%%%%%%%%%%%%%%%%%%%

\begin{frame}{\bf 2.3 A classes dos inteiros modulares}
\begin{block}{Ideias} 
\begin{enumerate}
\item Um {\it ideal} de um anel $R$ é um subgrupo aditivo $I$ de $R$ tal que $ab \in I$ para todo $a\in R$ e $b\in I$.
\item Dado um subconjunto $X\subseteq R$ o menor ideal $I$ de $R$ que contém $X$ chama-se o {\it ideal gerado por $X$} e é denotamos por $I=\langle X \rangle$.\pause
\item Dado um elmento $a\in R$ os múltiplos de $a$ formam um ideal de $R$ que é igual ao ideal gerado por $\{x\}$.
$$ \langle a \rangle = Ra = \{ ba  \mid b \in R\}.$$\pause
\item O ideal  $I=\langle a_1, \ldots, a_n\rangle =\left\{ \sum_{i=1}^n b_ia_i \mid b_1, \ldots, b_n \in R\right\}.$
\end{enumerate}
\end{block}
\end{frame}
 
 \begin{frame}
 \begin{block}{Exemplos}
 \begin{enumerate}
  \item Ideais de $\ZZ$ são da forma $n\ZZ  = \{ an \mid a \in \ZZ\}$.\pause
    \item Ideais de $\RR[x]$ são da forma $ \langle f(x) \rangle = \{g(x)f(x) \mid g(x)\in \RR[x]\}$.\pause
  \item O subconjunto $\{2f(x) + xg(x) \mid f(x),g(x) \in \ZZ[x]\}$ é um ideal de $\ZZ[x]$ que não é gerado por só um elemento.
 \end{enumerate}
 \end{block}
 \end{frame}
 
 
\begin{frame}{\bf 2.3 A classes dos inteiros modulares}
\begin{block}{Classes laterais}
\begin{enumerate}
\item A classe lateral à esquerda $a+I$ de um elemento $a\in R$ módulo um ideal $I$ é o subconjunto 
 $$ \overline{a}=a + I := \{ a + b | b\in I\} \subseteq R$$ \pause
 \item Para $a,b \in R$ tem-se que 
 $$a+I = b+I \Leftrightarrow a-b \in I.$$
 \end{enumerate}
 \end{block}
 \end{frame}
 
 \begin{frame}{}
 \begin{block}{Exemplos}
 \begin{enumerate}
  \item Para $m\in \ZZ$ a classe lateral de $a$ módulo $\ZZ m$ é 
  $$\overline{a}=a + m\ZZ  = \{\mbox{ os números cujo resto da divisão por $m$ é $a\% m$}\}.$$\pause
  Para $m=2$ só temos duas classes.
  $$ \overline{0}=0 + 2\ZZ = \{ \mbox { os números pares }\},$$
  $$ \overline{1}=1 + 2\ZZ = \{ \mbox { os números ímpares }\}.$$\pause
  \item Em geral existem exactamente $m$ classes laterais módulo $m$.

 \end{enumerate}
\end{block}
\end{frame}

 \begin{frame}{}
 \begin{block}{Exemplos}
 \begin{enumerate}
  \item Para $f(x) \in \RR[x]$ 
    $$g(x) + \langle f(x) \rangle  = \left\{ 
    \begin{array}{l} \mbox{ os polinómios cujo resto da divisão por $f(x)$}\\ \mbox{ é $g(x)\% f(x)$} \end{array} \right\}.$$\pause
  \item Para $f(x)=x^2+1\in \RR[x]$ temos que qualquer classe lateral tem a forma
  $$ a + bx + \langle x^2+1\rangle $$
  onde $a,b \in \RR$.
  
 \end{enumerate}
 \end{block}
 \end{frame}

 
\begin{frame}{\bf 2.3 A classes dos inteiros modulares}
\begin{block}{Anel quociente}
\begin{enumerate} 
\item  O anel quociente $R/I$ de $R$ módulo o ideal $I$ é o conjunto das classes laterais:
 $$R/I = \{ a+I | a\in R \}.$$\pause
\item As operações em $R/I$ são:
$$ (a+I) + (b+I) = (a+b) + I $$
$$ (a+I) \cdot (b+I) = (ab) + I,$$
para $a,b \in I$.
\end{enumerate}
\end{block}
\end{frame}


 \begin{frame}{}
 \begin{block}{Exemplos}
 \begin{enumerate}
  \item Para $m\in \ZZ$, tem-se que $$\ZZ/ m \ZZ = \ZZ_m = \{ \overline{0}, \overline{1}, \ldots, \overline{m-1}\}.$$\pause
  \item Para $f(x)\in \RR[x]$ não-nulo, o anel quociente $\RR[x]/\langle f(x) \rangle$ tem dimensão $n=\mathrm{grau}(f(x))$ e base 
  $ \{ \overline{1}, \overline{x}, \ldots, \overline{x^{n-1}}\}$  como espaço vectorial sobre $\RR$.\pause
  \item Para $f(x)=x^2+1$ tem-se que
  $$ \RR[x]/\langle x^2+1\rangle \simeq \CC \qquad \overline{a+bx} \mapsto a+\imath b.$$\pause
  \item Para $f(x)=x^2+1 \in \ZZ[x]$ tem-se que 
  $$ \ZZ[x]/\langle x^2+1\rangle \simeq \ZZ[\imath].$$
 \end{enumerate}
 \end{block}
 \end{frame}



\begin{frame}{\bf 2.3 A classes dos inteiros modulares}
\begin{block}{Teorema}
Seja $R$ um domínio Euclideano e $I=\langle m \rangle$ o ideal gerado por um elemento $m\in R$.
Para qualquer elemento $a\in R$ tem-se que 
$a+\langle m \rangle$ é invertível em $R/\langle m \rangle$ se e só se $\mdc{a}{m} \sim 1$
\end{block}
\end{frame}


 \begin{frame}{}
 \begin{block}{Definição}
Um elemento não invertível $p\in R$ diz-se irredutível (ou primo) se $a\mid p$ então $a\sim 1$ ou $a\sim p$.
 \end{block}\pause

 \begin{block}{Corolário}
Seja $R$ um domínio Euclideano. Então $\overline{R}=R/\langle m \rangle$ é um corpo se e só se $m$ é irredutível.
 \end{block}\pause
 
\medskip
{\bf Prova:}
Se $m$ for irredutível e $a\in R$ então $\mdc{a}{m} \sim 1$ ou $\mdc{a}{m} \sim p$. 
Se $\mdc{a}{m}\sim m$, então $m|a$ e $\overline{a} = \overline{0}$.

 \end{frame}

\begin{frame}{}
\begin{enumerate}
\item $\ZZ_m$ corpo se e só se $m$ é primo.\pause
\item $K[x]/\langle f(x)\rangle$ é corpo se e só se $f(x)$ é um polinómio irredutível.\pause
\item $\ZZ_2[x]/\langle x^2+x+1\rangle$ é corpo com $4$ elementos!\pause
\item {\bf Galois:} para qualquer $n \geq 1$ e número primo $p$ existe um polinómio irredutível $f(x)\in \ZZ_p[x]$ de grau $n$. Logo existe um corpo com $p^n$ elementos.
\end{enumerate}
\end{frame}


\begin{frame}
Como calcular os inversos em $R/\langle m \rangle$ ?\pause
\begin{block}{Inversos}
Sejam $a,m \in R$ com $\mdc{a}{m} \sim 1$.\\
Pelo Algoritmo estendido de Euclides existem $x,y \in R$ tais que 
$$ 1 = xa + ym \qquad \mbox{ou seja } \qquad 1-xa \in \langle m \rangle \Leftrightarrow \overline{1}=\overline{x}\overline{a}.$$ 
Logo $\overline{x}$ é o inverso de $\overline{a}$ em $R/\langle m \rangle$.
\end{block}
\end{frame}

\begin{frame}{\bf Equações Diofantinas Lineares}
A equação diofantinas lineares:
$$ a_1x_1 + a_2x_2 + \ldots + a_n x_n = c,$$
com $a_1, \ldots, a_n, c \in \ZZ$ tem soluções $(x_1, \ldots, x_n)\in \ZZ^n$ se e só se $\mathrm{mdc}(a_1,\ldots, a_n) | c$.
\end{frame}

\begin{frame}{\bf Equações Diofantinas Lineares}
$$ax+by=c\qquad \Leftrightarrow \qquad a'x+b'y = c'$$
onde $a'=\frac{a}{\mdc{a}{b}}, b'=\frac{b}{\mdc{a}{b}}, c'=\frac{c}{\mdc{a}{b}}$.

Existem $s,t\in \ZZ$ tais que $1=sa'+tb'$ e $c'=c'sa'+c'tb'$.

\begin{eqnarray*}
&\Rightarrow &a'(x-c's)+b'(y-c't) = c'-c'=0\\
&\Leftrightarrow& a'(x-c's)=b'(c't-y) \\
&\Rightarrow& a' | (c't-y) \qquad \mbox{ porque } \mdc{a'}{b'}=1\\
&\Rightarrow& c't-y = a'n, \qquad n\in \ZZ \\
&\Rightarrow& y = c't-a'n, \qquad n\in \ZZ \\
&\Rightarrow& x = c't+b'n, \qquad n\in \ZZ
\end{eqnarray*}
\end{frame}

\begin{frame}[fragile]{A classe dos inteiros módulares}
\lstset{language=C++}
\begin{lstlisting}

class modular
{
	unsigned int modulo;
	int numero;
	
	int mdc(int,int);
	
	public: 
		modular(unsigned int,int);
		modular operator + (modular);
		modular operator - (modular);
		modular operator * (modular);
		bool invertivel();
		modulear inverso ();
}
\end{lstlisting}
\end{frame}

\begin{frame}[fragile]{A classe dos inteiros módulares}
\lstset{language=C++,basicstyle=\tiny}
\begin{lstlisting}
modular(unsigned int n, int a)
{
	base=n;
	numero=a%n;
};

modular modular::operator + (modular b)
{
	if (b.base == base) 
		return modular(base,b.numero+numero)
};

bool invertivel()
{
	return (mdc(numero,base)==1);
};

\end{lstlisting}
\end{frame}

\begin{frame}[fragile]{Algoritmo de Euclid em C++}
\lstset{language=C++,basicstyle=\tiny}
\begin{lstlisting}
int mdc(int a, int b, int* x, int* y)
{
	int r0,s0,t0,r1,s1,t1;
	r0=a;	s0=1;	t0=0;
	r1=b;	s1=0;	t1=1;
	
	while(r1!=0)
	{
	  int q = r0/r1;	  
	  int h = r0%r1; r0=r1; r1=h;
	  h=s0-q*s1; s0=s1; s1=h;
	  h=t0-q*t1; t0=t1; t1=h;
	};
	(*x)=s0;
	(*y)=t0;
	return r0;
};

int main()
{
		int a,b,x,y;
		x=0;
		y=0;
		cin >> a >> b;
		cout << "mdc=" << mdc(a,b,&x,&y);
		cout << " = " << x << "*" << a << " + "<<  y << "*" << b << endl;
}
\end{lstlisting}
\end{frame}


\end{document}





2.2 A classes dos números racionais
2.3 A classes dos inteiros modulares
2.4 O template dos polinómios.
2.5 Multiplicação de polinómios: Algoritmo de Karatsuba

3. 
3.1 O template de vectores
3.2 O template de matrizes
3.3 Multiplicação rápida de matrizes: Algoritmo de Strassen.
3.4 Resolução de sistemas lineares

4. 
4.1 Introdução à geometria algébrica
4.2 Ordem nos monómios
4.3 Algoritmo da divisão para polinómios em múltiplos indeterminadas
4.4 Lemma de Dickson
4.5 Bases de Gröbner e Teorema de Hilbert
4.6 Algoritmo de Buchberger
4.7 Teoria de Eliminação
4.8 Teorema "Nullstellensatz" de Hilbert
4.9 Algoritmo de filiação de radicais
4.10 Aplicação à robótica
\end{frame}

