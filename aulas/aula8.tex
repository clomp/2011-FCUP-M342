%\documentclass[handout]{beamer}
\documentclass{beamer}
\usepackage{algorithmic}
\usepackage{polynom}
\usepackage[english, portuguese]{babel}
\usepackage{hyperref}
\usepackage{amssymb}
\usepackage[utf8x]{inputenc}
\usepackage{euler}

\usepackage[center]{caption}

\usepackage{listings}
\lstset{language=C++, basicstyle=\footnotesize}

%\floatname{algorithm}{Função }
\renewcommand{\algorithmicrequire}{\textbf{Input:}}
\renewcommand{\algorithmicensure}{\textbf{Output:}}
\newcommand{\q}{\mathbf{q}}
\newcommand{\ZZ}{\mathbb Z}
\newcommand{\NN}{\mathbb N}
\newcommand{\CC}{\mathbb C}
\newcommand{\RR}{\mathbb R}
\newcommand{\QQ}{\mathbb Q}
\newcommand{\grau}[1]{\mathrm{grau}({#1})}
\newcommand{\mdc}[2]{\mathrm{mdc}({#1}, {#2})}
\newcommand{\mmc}[2]{\mathrm{mmc}({#1}, {#2})}
\newcommand{\quo}[2]{{#1}\, \mathrm{quo} \, {#2}}
\newcommand{\rem}[2]{{#1}\, \mathrm{rem} \, {#2}}

\newcommand{\cvector}[1]{\mathrm{vector}(#1)}
\newcommand{\lc}[1]{\mathrm{lc}(#1)}
%\usetheme{Berlin}
\usetheme[pageofpages=of,% String used between the current page and the
                         % total page count.
          bullet=circle,% Use circles instead of squares for bullets.
          titleline=true,% Show a line below the frame title.
          alternativetitlepage=true,% Use the fancy title page.
          %titlepagelogo=logotipo_pdf1,% Logo for the first page.
          %watermark=logotipo_pdf1,% Watermark used in every page.
          %watermarkheight=100px,% Height of the watermark.
          %watermarkheightmult=2,% The watermark image is 4 times bigger
                                % than watermarkheight.
          ]{Torino}

\title[M342]{M342 Álgebra Computacional}
\author{Christian Lomp}
\institute{FCUP}
\date{17 de Outubro de 2011}
\begin{document}

\begin{frame}
\titlepage
\end{frame}


%3. 
%3.1 O template de vectores
%3.2 O template de matrizes
%3.3 Multiplicação rápida de matrizes: Algoritmo de Strassen.
%3.4 Resolução de sistemas lineares





\begin{frame}{\bf 3  Álgebra linear }{\bf Determinante}


Para calcular a determinante o mais rápido é usar o algoritmo de eliminação de Gauss para transformar uma matriz numa matriz em escada.


\begin{figure}[h]
  \begin{center}
    \includegraphics[scale=1]{Carl_Friedrich_Gauss.jpg}
  \end{center}

  \caption*{Johann Carl Friedrich Gauss (1777-1855)}
  \label{demTP}
\end{figure}

\end{frame}


\begin{frame}{\bf 3  Álgebra linear }
\begin{block}{Operações de linhas}
Seja $A=(a_{ij})_{1\leq i,j\leq n}$ cujas linhas denotamos por $L_1, \ldots, L_n$.\pause 

Dado um escalar $\lambda \in K$ e  $1\leq i \neq j \leq n$ a operação de linha
$$L_i \leftarrow L_i - \lambda L_j$$ substitue a $i$-esmia linha $L_i = (a_{i1}, \ldots, a_{in})$ de $A$ pela linha  $$L_i-\lambda L_j = (a_{i1}-\lambda a_{j1}, \ldots, a_{in}-\lambda a_{jn}).$$\pause
\end{block}
\end{frame}
\begin{frame}{\bf 3  Álgebra linear }
\begin{block}{Operações de linhas}

Se $E_{ij} = ( \delta_{ij} )_{1\leq i,j \leq n}$ é a matriz que tem entrada $1$ na componente $(i,j)$ e $0$ nos restantes componentes, então a multiplicação $E_{ij}A$ consiste de matriz cuja $i$-ésima linha é a $j$-ésima linha de $A$ e cujas outras linhas são nulos.\pause

O resultado da operação da linha $L_i \leftarrow L_i - \lambda L_j$ na matriz $A$ é 
$$ (Id_n - \lambda E_{ij})A.$$
\end{block}
\end{frame}
\begin{frame}{\bf 3  Álgebra linear }
\begin{block}{Eliminação de Gauss}

Se $E_{ij} = ( \delta_{ij} )_{1\leq i,j \leq n}$ é a matriz que tem entrada $1$ na componente $(i,j)$ e $0$ nos restantes componentes, então a multiplicação $E_{ij}A$ consiste de matriz cuja $i$-ésima linha é a $j$-ésima linha de $A$ e cujas outras linhas são nulos.\pause

O resultado da operação da linha $L_i \leftarrow L_i - \lambda L_j$ na matriz $A$ é 
$$ (Id_n - \lambda E_{ij})A.$$
\end{block}
\end{frame}


\begin{frame}[fragile]{\bf 3  Álgebra linear }
\begin{algorithmic}
\REQUIRE matriz $A$
\ENSURE matriz em escada $A'$
    \FOR{$1\leq j < n$}
      \IF {$\exists k: a_{kj}\neq 0$}
      	\STATE {trocar $L_k$ e $L_j$}
		\FOR{$j<k\leq n$}
			\STATE $L_k \rightarrow a_{kj}L_k-a_{jj}L_k$
		\ENDFOR
	  \ENDIF
	 \ENDFOR
\end{algorithmic}

$$\sum_{j=1}^{n-1} (n-j) = \frac{n(n-1)}{2} \:\:\mbox{ iterações.}$$
\end{frame}


\begin{frame}[fragile]{\bf 3  Álgebra linear }
\lstset{language=C++}
\begin{lstlisting}
   MATRIZ MATRIZ::OpLinha(int i, int j, Tipo lambda, Tipo mu)
   {
     vector<VETOR> output(data);                 
     for (int k=0; k<data.size(); k++) 
      
       output[i][k] = mu*output[i][k] - lambda*output[j][k];
     
     return MATRIZ(output);
   };
\end{lstlisting}
\end{frame}


\begin{frame}[fragile]{\bf 3  Álgebra linear }
\lstset{language=C++}
\begin{lstlisting}
   MATRIZ MATRIZ::Trocar(int i, int j)
   {
     vector<VETOR> output(data);                 
     for (int k=0; k<data.size(); k++) 
     {      
       Tipo aux=output[i][k];
       output[i][k]=output[j][k];
       output[j][k]=aux;
     }
     return MATRIZ(output);
   };
\end{lstlisting}
\end{frame}


\begin{frame}[fragile]{\bf 3  Álgebra linear }
\lstset{language=C++,basicstyle=\tiny}
\begin{lstlisting}
   MATRIZ MATRIZ::TransformarEscadas()
   {
     int n=data.size();
     MATRIZ output(data);
     for (int j=0; j<n; j++) 
     {      
       int k=j;
       while(k<n && output.data[k][j] == 0) k++;
       if (k<n) 
       {	
          output=output.trocar(j,k);
          for (k=j+1; k<n; k++)
            output=output.OpLinha(k,j,output.data[k][j],output.data[j][j]);
       }
     };
     return output;
   };
\end{lstlisting}
\end{frame}


\begin{frame}[fragile]{\bf 3  Álgebra linear }
\lstset{language=C++}
\begin{lstlisting}
   Tipo MATRIZ::Determinante()
   {
     MATRIZ escadas=TransformarEscadas();
     Tipo det=escadas.data[0][0];
     for(int i=1: i<escadas.data.size(); i++)
     	det=det*escadas.data[i][i];
     return det;
   };
\end{lstlisting}
\end{frame}

\begin{frame}{\bf 3.3 Algoritmo de Strassen}
Multiplicação rápida de matrizes.
\begin{figure}[h]
  \begin{center}
    \includegraphics[scale=0.4]{Volker_Strassen.png}
  \end{center}

  \caption*{Volker Strassen (1936-)}
  \label{demTP}
\end{figure}

\end{frame}


\begin{frame}{\bf 3.3 Algoritmo de Strassen}
$$ M=\left(\begin{array}{cc} A & B \\ C & D \end{array}\right) \in M_{2^n}(K)\qquad 
N=\left(\begin{array}{cc} E & F \\ G & H \end{array}\right)\in M_{2^n}(K) $$
$A,B,C,D,E,F,G,H \in M_{2^{n-1}}(K)$. Então 
$$ MN=\left(\begin{array}{cc} R & S \\ T & U \end{array}\right) \qquad 
\mbox{ tal que } \qquad \left\{ \begin{array}{rcl} R & = & AE + BG \\
S & = & AF + BH \\
T & = & CE + DG \\
U & = & CF + DH  \end{array}\right.$$

$$ 8 \mbox{ Multiplicações } \qquad 4 \mbox{ Adições}$$
\end{frame}


\begin{frame}{\bf 3.3 Algoritmo de Strassen}

$$\left\{\begin{array}{rcl}
    M_{1} &=& (A + D)\cdot (D + H)\\
    M_{2} &=& (C + D)\cdot E\\
    M_{3} &=& A \cdot(F - H)\\
    M_{4} &=& D \cdot(G - E)\\
    M_{5} &=& (A + B)\cdot H\\
    M_{6} &=& (C - A) \cdot(E + F)\\
    M_{7} &=& (B - D) \cdot(G + H) 
    \end{array}\right. \qquad 
    \left\{ \begin{array}{rcl} 
    R & = & M_1 + M_4 - M_5 + M_7 \\
S & = & M_3 + M_5 \\
T & = & M_2+M_4 \\
U & = & M_1-M_2+M_3+M_6  \end{array}\right.$$

$$ 7 \mbox{ Multiplicações } \qquad 18 \mbox{ Adições}$$
\end{frame}

\begin{frame}{\bf 3.3 Algoritmo de Strassen}
\begin{block}{Caso 1 do ``Master Theorem''}
Dado uma função recursiva que divide os dados da entrada com tamanho $n$ em $b$ subconjuntos e que aplica recursivamente essa função $a$ vezes nestes subconjuntos e que tem custos adicionais dado por uma função $f(n)$ satisfaz a formula de custos:
$$T(n)=aT(n/b)+f(n)$$
Se $f(n) \in O(n^{\log_b(a) - \epsilon})$ onde $\epsilon >0$ então $T(n) \in \Theta(n^{\log_b(a)})$. 
\end{block}

\begin{block}{Notação O}
$$f(x)\in O(g(x)) \Leftrightarrow \exists C, n_0: \forall n>n_0: f(n)\leq C g(n).$$
$$f(x)\in \Theta(g(x)) \Leftrightarrow \exists C_1,C_2, n_0: \forall n>n_0: C_1g(n)\leq f(n)\leq C_2g(n).$$
\end{block}
\end{frame}

\begin{frame}{\bf 3.3 Algoritmo de Strassen}
\begin{block}{Multiplicação de matrizes usual}
Tamanho das matrizes $n\times n$:
$$T(n) = 8 T(n/2) + 4n^2$$
Como $f(n) = 4n^2 \in O(n^{\log_2(8) - 1})$ então $T(n) \in \Theta(n^{\log_2(8)})=\Theta(n^3)$. 
\end{block}

\begin{block}{Multiplicação de matrizes usando o Algoritmo de Strassen}
Tamanho das matrizes $n\times n$:
$$T(n) = 7 T(n/2) + 18n^2$$
Como $f(n) = 18n^2 \in O(n^{\log_2(7)-0.8})$ então $T(n) \in \Theta(n^{2.807})$, onde $\log_2(7) \simeq 2.807$.
\end{block}

\end{frame}

\begin{frame}{\bf 3.3 Algoritmo de Strassen}

\begin{tabular}{|c|r|r|r|}\hline
$n=2^k$ & Classico $8^k$& Strassen $7^k$ & percentagem \\\hline
$2$ & $8$ & $7$ & $\sim 87\%$\\\hline
$4$ & $64$ & $49$ & $\sim 76\%$ \\\hline
$8$ & $512$ & $343$ & $\sim 67\%$\\\hline
$16$ & $4096$ & $2401$ & $\sim 58\%$\\\hline
$32$ & $32768$ & $16807$ & $\sim 51\%$\\\hline
$64$ & $262144$ & $117649$ & $\sim 49\%$ \\\hline
$128$ & $2097152$ & $823543$ & $\sim 39\%$\\\hline
$256$ & $16777216$ & $5764801$ & $\sim 34\%$\\\hline
$512$ & $134217728$ & $40353607$ & $\sim 30\%$\\\hline
$1024$ & $1073741824$ & $282475249$ & $\sim 26\%$\\\hline
\end{tabular}
\end{frame}



\end{document}





3. 
3.1 O template de vectores
3.2 O template de matrizes
3.3 Multiplicação rápida de matrizes: Algoritmo de Strassen.
3.4 Resolução de sistemas lineares

4. 
4.1 Introdução à geometria algébrica
4.2 Ordem nos monómios
4.3 Algoritmo da divisão para polinómios em múltiplos indeterminadas
4.4 Lemma de Dickson
4.5 Bases de Gröbner e Teorema de Hilbert
4.6 Algoritmo de Buchberger
4.7 Teoria de Eliminação
4.8 Teorema "Nullstellensatz" de Hilbert
4.9 Algoritmo de filiação de radicais
4.10 Aplicação à robótica
\end{frame}

