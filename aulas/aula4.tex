\documentclass[handout]{beamer}
\usepackage{algorithmic}
\usepackage{polynom}
\usepackage[english, portuguese]{babel}
\usepackage{hyperref}
\usepackage{amssymb}
\usepackage[utf8x]{inputenc}
\usepackage{euler}
\usepackage{listings}
\lstset{language=C++, basicstyle=\footnotesize}

%\floatname{algorithm}{Função }
\renewcommand{\algorithmicrequire}{\textbf{Input:}}
\renewcommand{\algorithmicensure}{\textbf{Output:}}
\newcommand{\q}{\mathbf{q}}
\newcommand{\ZZ}{\mathbb Z}
\newcommand{\NN}{\mathbb N}
\newcommand{\CC}{\mathbb C}
\newcommand{\QQ}{\mathbb Q}
\newcommand{\grau}[1]{\mathrm{grau}({#1})}
\newcommand{\mdc}[2]{\mathrm{mdc}({#1}, {#2})}
\newcommand{\mmc}[2]{\mathrm{mmc}({#1}, {#2})}
\newcommand{\quo}[2]{{#1}\, \mathrm{quo} \, {#2}}
\newcommand{\rem}[2]{{#1}\, \mathrm{rem} \, {#2}}

\newcommand{\cvector}[1]{\mathrm{vector}(#1)}
\newcommand{\lc}[1]{\mathrm{lc}(#1)}
%\usetheme{Berlin}
\usetheme[pageofpages=of,% String used between the current page and the
                         % total page count.
          bullet=circle,% Use circles instead of squares for bullets.
          titleline=true,% Show a line below the frame title.
          alternativetitlepage=true,% Use the fancy title page.
          %titlepagelogo=logotipo_pdf1,% Logo for the first page.
          %watermark=logotipo_pdf1,% Watermark used in every page.
          %watermarkheight=100px,% Height of the watermark.
          %watermarkheightmult=2,% The watermark image is 4 times bigger
                                % than watermarkheight.
          ]{Torino}

\title[M342]{M342 Álgebra Computacional}
\author{Christian Lomp}
\institute{FCUP}
\date{21 de setembro de 2011}
\begin{document}

\begin{frame}
\titlepage
\end{frame}






\begin{frame}{\bf 2.2 Os numeros racionais}{\bf {o algoritmo de Euclid}}
\begin{enumerate}
\item O algoritmo de Euclid para inteiros permite determinar o {\bf máximo divisor comum} de dois inteiros.\pause
\item O algoritmo de Euclid permite reduzier quocientes de inteiros $\frac{a}{b} = \frac{a'}{b'}$ onde $\mdc{a'}{b'}=1$.\pause
\item O algotitmo aplica-se também para polinómios.
\end{enumerate}
\end{frame}

\begin{frame}
\begin{block}{Definição} Um domínio integral $R$ com uma função $d:R\rightarrow \NN\cup\{-\infty\}$  chama-se um {\bf Domínio Euclideano} se 
$$\forall a,b \in R \mbox{ com } b\neq 0: \qquad \exists q,r \in R \: a=qb+r \:\mbox{ com } d(r)<d(b).$$
\pause
O elemento $q$, denotado por $q=a/b$ ou $q=\quo{a}{b}$, diz-se o {\bf quociente} e $r$, denotado por $r=a\%b$ ou $q=\rem{a}{b}$ diz-se o {\bf resto}. 

\pause Em geral $a/b$ e $a\%b$ não são necessáriamente únicos.
\end{block}
\end{frame}

\begin{frame}
\begin{block}{Exemplos}
\begin{enumerate}
\item $R=\ZZ$ e $d:\ZZ \rightarrow \NN$ é dado por $d(a)=|a|, \: \forall a\in \ZZ$.\pause
\item $R=F[x]$, $F$ um corpo e $d:F[x] \rightarrow \NN\cup \{-\infty\}$  dado por $d(f)=\grau{f}$ se $f\neq 0$ e $\grau{0}=-\infty$.\pause
\item $R=\ZZ[i] = \{a+\imath b\in\CC \mid a,b\in\ZZ\}$, os inteiros de Gauss, e $d(a+\imath b) = a^2+b^2$.\pause
\item $R=F$ um corpo e $d(a)=1$ se $a\neq 0$ e $d(a)=0$ se $a=0$.
\end{enumerate}
\end{block}
\end{frame}

\begin{frame}
Se $d(b)=-\infty$ então $b=0$, pois se $d(b)=-\infty$ e $b\neq 0$, então existem $q,r \in R$ tais que 
$$ b=qb+r \qquad \mbox{ com } d(r) < d(b)=-\infty$$
o que é impossível, pois $d(r)\geq -\infty$ para qualquer $r\in R$.
\end{frame}

\begin{frame}
\begin{block}{Definição}
Sejam $a$ e $b$ elementos num domínio de integridade $R$. Digamos que $a$ {\bf divide} $b$, denotado por $a|b$ se e só se 
$$ \exists c\in R: \qquad b=ac.$$
\end{block}
\pause

Exercício: Sejam $a,b$ elementos de um domínio de integridade $R$.
\begin{enumerate}
\item $0|a$ se e só se $a=0$;\pause
\item $a|1$ se e só se $a$ é invertível;\pause
\item $a|b$ e $b|a$ se e só se existe um elemento invertível $u\in R$ tais que $a=ub$.\pause
\end{enumerate}

\begin{block}{Definição}
Dois elementos $a$ e $b$ chamam-se {\bf associados} se $a=ub$ para um elemento invertível $u\in R$. Denotamos este facto por $a\sim b$.
\end{block}
\end{frame}


\begin{frame}
\begin{block}{Definição}
Sejam $a,b,c$ elementos dum domínio de integridade $R$. O elemento $c$ diz-se {\bf máximo divisor comum} de $a$ e $b$, denotado por $c=\mdc{a}{b}$ se 
\begin{enumerate}
\item[(i)] $c|a$ e $c|b$;
\item[(ii)] se $d|a$ e $d|b$ então $d|c$, para todo $d\in R$.
\end{enumerate}

\pause O elemento $c$ diz-se {\bf mínimo múltiplo comum} de $a$ e $b$, denotado por $c=\mmc{a}{b}$ se 
\begin{enumerate}
\item[(i)] $a|c$ e $b|c$;
\item[(ii)] se $a|d$ e $b|d$ então $c|d$, para todo $d\in R$.
\end{enumerate}\pause
Em $\ZZ$ define-se que $\mdc{a}{b}$ e $\mmc{a}{b}$ sejam também positivos e portanto únicos.

\end{block}

\end{frame}


\begin{frame}
\begin{block}{Propriedades de $\mdc{-}{-}$ em $\ZZ$} Sejam $a,b,c \in \ZZ$: 
\begin{enumerate}
\item[(i)] $\mdc{a}{b}=|a| \Longleftrightarrow a | b$;
\item[(ii)] $\mdc{a}{a}=\mdc{a}{0}=|a|$ e $\mdc{a}{1}=1$.
\item[(iii)] $\mdc{a}{b}=\mdc{b}{a}$ (comutatividade);
\item[(iv)] $\mdc{a}{\mdc{b}{c}} = \mdc{\mdc{a}{b}}{c}$ (associatividade);
\item[(v)]  $\mdc{ca}{cb} = |c|\mdc{a}{b}$ (distributividade);
\item[(vi)] $|a|=|b| \Rightarrow \mdc{a}{c}=\mdc{b}{c}$.
\item[(vii)] $\mdc{a}{b} = \mdc{b}{a\%b}$
\end{enumerate}
\end{block}
\end{frame}

\begin{frame}[fragile]{Algoritmo de Euclid}
\begin{algorithmic}
\REQUIRE $a,b \in R$ onde $R$ é um domínio de Euclid.
\ENSURE Um máximo divisor comum $c$ de $a$ e $b$;
	\STATE $r_0 \leftarrow a$
	\STATE $r_1 \leftarrow b$
	\STATE $i \leftarrow 1$
	\WHILE{$r_i\neq 0$}
		\STATE $r_{i+1}\leftarrow (r_{i-1}\% r_i)$
		\STATE $i\leftarrow i+1$
	\ENDWHILE
	\RETURN $r_i-1$
\end{algorithmic}

\end{frame}


\begin{frame}[fragile]{Algoritmo de Euclid em C++}
\lstset{language=C++,basicstyle=\tiny}
\begin{lstlisting}
int mdc(int a, int b)
{
  int r0=a;
  int r1=b;
  while (r1!= 0)
  {
    int aux=r1;
    r1=r0%r1;
    r0=aux; 
  }
  return r0;
}
\end{lstlisting}
\pause

\begin{lstlisting}[language=C++]
int mdc(int a, int b)
{
  if (b==0) 
    return a;
  else 
    return mdc(b,a%b)  ;
};
\end{lstlisting}
\end{frame}


\begin{frame}{Algoritmo de Euclid estendido}
O máximo divisior comum $\mdc{a}{b}$ de dois inteiro é combinação linear de $a$ e $b$.
\begin{block}{Exemplo}
\begin{eqnarray*}
126 &=& 3\times 35 + 21 \\
35 &=& 1\times 21 + 14 \\
21 &=& 1 \times 14 + 7\\
14 &=& 2\times 7 + 0
\end{eqnarray*}
{\tiny $$7 = 21 - 1\times 14 = 21 - 1\times (35 - 1\times 21) = 2\times (126 - 3\times 35) - 1\times 35 = 2\times 126 + (-7)\times 35.$$}
Logo $\mdc{126}{35} = 7 = 2\times 126 + (-7)\times 35$.
\end{block}
\end{frame}

\begin{frame}[fragile]{Algoritmo de Euclid estendido}
\begin{algorithmic}
\REQUIRE $a,b \in R$ onde $R$ é um domínio de Euclid.
\ENSURE $l\in \NN, r_i,s_i,t_i \in R$ para $0\leq i\leq l+1$ e $q_i\in R$ para $1\leq i \leq l$.
	\STATE $r_0 \leftarrow a$, $s_0 \leftarrow 1$, $t_0\leftarrow 0$
	\STATE $r_1 \leftarrow b$, $s_1 \leftarrow 0$, $t_1\leftarrow 1$
	\STATE $i \leftarrow 1$
	\WHILE{$r_i\neq 0$}
		\STATE $q_i \leftarrow (r_{i-1}/ r_i)$
		\STATE $r_{i+1}\leftarrow (r_{i-1}\% r_i)$
		\STATE $s_{i+1}\leftarrow s_{i-1} - q_is_i$
		\STATE $t_{i+1}\leftarrow t_{i-1} - q_it_i$
		\STATE $i\leftarrow i+1$
	\ENDWHILE
	\STATE $l\leftarrow i-1$
	\RETURN $l, r_i, s_i, t_i$ para $0\leq i \leq l+1$ e $q_i$ para $1\leq i \leq l$.
\end{algorithmic}

\end{frame}


\begin{frame}{}
\begin{block}{Exemplo}
$R=\ZZ$, $a=126$ e $b=35$:
$$\begin{array}{|r|r|r|r|r|}\hline
i & q_i & r_i & s_i & t_i \\\hline
0 &   & 126 & 1 &0 \\
1 & 3 & 35 & 0 &1 \\
2 & 1 & 21 & 1 & -3 \\
3 & 1 & 14 & -1 & 4 \\
4 &  2 & 7 & {\bf 2} & {\bf -7} \\
5 &   & 0 & -5 & 18 \\\hline
\end{array}$$
\end{block}
$$7 = 2\times 126 + (-7)\times 35.$$
\end{frame}

\begin{frame}{}
\begin{block}{Exemplo}
$R=\QQ[x]$, $a=18x^3-42x^2+30x-6$ e $b=-12x^2+10x-2$:
$$\begin{array}{|r|r|r|r|r|}\hline
i & q_i                         &               r_i         & s_i                        & t_i \\\hline
0 &                             & 18x^3-42x^2+30x-6         & 1                          & 0 \\
1 & -\frac{3}{2}x + \frac{9}{4} &      -12x^2+10x-2         & 0                          & 1 \\
2 & -\frac{8}{3}x + \frac{4}{3} &  \frac{9}{2}x-\frac{3}{2} &{\bf 1}                     & {\bf \frac{3}{2}x-\frac{9}{4}} \\
3 &                             &                         0 & \frac{8}{3}x - \frac{4}{3} & 4x^2-8x+4 \\\hline 
\end{array}$$
\end{block}
$$ \boxed{\frac{9}{2}x-\frac{3}{2}  = 18x^3-42x^2+30x-6 +  \left(\frac{3}{2}x-\frac{9}{4}\right)\left(-12x^2+10x-2\right)  }$$
\end{frame}


\end{document}





2.2 A classes dos números racionais
2.3 A classes dos inteiros modulares
2.4 O template dos polinómios.
2.5 Multiplicação de polinómios: Algoritmo de Karatsuba

3. 
3.1 O template de vectores
3.2 O template de matrizes
3.3 Multiplicação rápida de matrizes: Algoritmo de Strassen.
3.4 Resolução de sistemas lineares

4. 
4.1 Introdução à geometria algébrica
4.2 Ordem nos monómios
4.3 Algoritmo da divisão para polinómios em múltiplos indeterminadas
4.4 Lemma de Dickson
4.5 Bases de Gröbner e Teorema de Hilbert
4.6 Algoritmo de Buchberger
4.7 Teoria de Eliminação
4.8 Teorema "Nullstellensatz" de Hilbert
4.9 Algoritmo de filiação de radicais
4.10 Aplicação à robótica
\end{frame}

