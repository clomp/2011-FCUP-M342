%\documentclass[handout]{beamer}
\documentclass{beamer}
\usepackage{algorithmic}
\usepackage{polynom}
\usepackage[english, portuguese]{babel}
\usepackage{hyperref}
\usepackage{amssymb}
\usepackage[utf8x]{inputenc}
\usepackage{euler}

\usepackage[center]{caption}

\usepackage{listings}
\lstset{language=C++, basicstyle=\footnotesize}

%\floatname{algorithm}{Função }
\renewcommand{\algorithmicrequire}{\textbf{Input:}}
\renewcommand{\algorithmicensure}{\textbf{Output:}}
\newcommand{\q}{\mathbf{q}}
\newcommand{\ZZ}{\mathbb Z}
\newcommand{\NN}{\mathbb N}
\newcommand{\CC}{\mathbb C}
\newcommand{\RR}{\mathbb R}
\newcommand{\QQ}{\mathbb Q}
\newcommand{\grau}[1]{\mathrm{grau}({#1})}
\newcommand{\mdc}[2]{\mathrm{mdc}({#1}, {#2})}
\newcommand{\mmc}[2]{\mathrm{mmc}({#1}, {#2})}
\newcommand{\quo}[2]{{#1}\, \mathrm{quo} \, {#2}}
\newcommand{\rem}[2]{{#1}\, \mathrm{rem} \, {#2}}

\newcommand{\cvector}[1]{\mathrm{vector}(#1)}
\newcommand{\lc}[1]{\mathrm{lc}(#1)}
%\usetheme{Berlin}
\usetheme[pageofpages=of,% String used between the current page and the
                         % total page count.
          bullet=circle,% Use circles instead of squares for bullets.
          titleline=true,% Show a line below the frame title.
          alternativetitlepage=true,% Use the fancy title page.
          %titlepagelogo=logotipo_pdf1,% Logo for the first page.
          %watermark=logotipo_pdf1,% Watermark used in every page.
          %watermarkheight=100px,% Height of the watermark.
          %watermarkheightmult=2,% The watermark image is 4 times bigger
                                % than watermarkheight.
          ]{Torino}

\title[M342]{M342 Álgebra Computacional}
\author{Christian Lomp}
\institute{FCUP}
\date{24 de Outubro de 2011}
\begin{document}

\begin{frame}
\titlepage
\end{frame}


%3. 
%3.1 O template de vectores
%3.2 O template de matrizes
%3.3 Multiplicação rápida de matrizes: Algoritmo de Strassen.
%3.4 Resolução de sistemas lineares





\begin{frame}{\bf 3.4 Resolução de sistemas lineares}


$$\left\{ \begin{array}{cccccccc}
a_{11}x_1 &+& a_{12}x_2 & \ldots &+& a_{1n}x_n &=& b_1 \\
a_{21}x_1 &+& a_{22}x_2 & \ldots &+& a_{2n}x_n &=& b_2 \\
   \vdots &&           & \ddots &&           & &\vdots \\
a_{m1}x_1 &+& a_{m2}x_2 & \ldots &+& a_{mn}x_n &=& b_m \\
\end{array}\right.$$
\end{frame}


\begin{frame}{\bf 3.4 Resolução de sistemas lineares }{Stratégia}
{\small 
$$
\left( \begin{array}{cccc|l}
a_{11} & a_{12} & \ldots & a_{1n} & b_1 \\
a_{21} & a_{22} & \ldots & a_{2n} & b_2 \\
   \vdots &     & \ddots &      &\vdots \\
a_{m1} & a_{m2} & \ldots & a_{mn} & b_m \\
\end{array}\right)\rightarrow
\left( \begin{array}{cccccc|l}
a'_{11}   & a'_{12} & \ldots  & \ldots  & \ldots  & a'_{1n}  & b'_1 \\
0         & a'_{22} & \ldots  &   &   & a'_{2n}  & b'_2 \\
   \vdots & \ddots  & \ddots  &   &   &  \vdots        & \vdots \\
 0        & \ldots  &  0      & a'_{kk} & \ldots  & a'_{kn}  & b'_k \\
\vdots   &         & \vdots  & 0       & \ldots  & 0        & b'_{k+1} \\
         &         &         &  \vdots       &   &   \vdots & \vdots \\
0         & \ldots  & 0      &  0       &  \ldots       & 0         & b'_m \\
\end{array}\right)
$$}
\end{frame}

\begin{frame}
Para obter uma matriz desta forma é as vezes necessário trocar colunas ou seja é necessário permutar os variáveis.

\medskip
\pause
Seja $\pi \in S_n$ a permutação tal que a $i$-esima coluna corresponde a variável $x_{\pi(i)}$.
\end{frame}


\begin{frame}{\bf 3.4}{}
\begin{block}{Sistemas impossíveis}
Se $(b_{k+1}, \ldots, b_m) \neq (0,\ldots, 0)$ então o sistema é impossível.
\end{block}
\pause

Segunda Redução pelo operações de linhas:

{\tiny 
$$
\left( \begin{array}{cccccc|l}
a'_{11}   & a'_{12} & \ldots  & \ldots  & \ldots  & a'_{1n}  & b'_1 \\
0         & a'_{22} &  &   &   & a'_{2n}  & b'_2 \\
   \vdots & \ddots  & \ddots  &   &   &  \vdots        & \vdots \\
 \vdots       &  &  0      & a'_{kk} & \ldots  & a'_{kn}  & b'_k \\
\vdots   &         & \vdots  & 0       & \ldots  & 0        & b'_{k+1} \\
         &         &         &  \vdots       &   &   \vdots & \vdots \\
0         & \ldots  & 0      &  0       &  \ldots       & 0         & b'_m \\
\end{array}\right)
\rightarrow
\left( \begin{array}{ccccccc|l}
1   & 0       & \ldots  & 0       & a''_{1 {k+1}}& \ldots  & a''_{1n}  & b''_1 \\
0         & 1 &      \ddots  & \vdots       & a''_{2 {k+1}}& \ldots  & a''_{2n}  & b''_2 \\
   \vdots & \ddots  & \ddots  &  0           & \vdots        &         &  \vdots   & \vdots \\
 0        & \ldots  &  0      & 1      & a''_{k {k+1}}   & \ldots  & a''_{kn}  & b''_k \\
\vdots   &         & \vdots   & 0            & \ldots            &\ldots   & 0        & 0 \\
         &         &         &  \vdots       &               &         & \vdots & \vdots \\
0         & \ldots  & 0      &  0            & \ldots             &  \ldots  & 0         & 0 \\
\end{array}\right)
$$}


\end{frame}

\begin{frame}{\bf 3.4}{Possível determinado}

\begin{block}{Sistemas possíveis determinados}
Se o sistema for possível e se $k=n$ então o sistema é {\bf possível determinado} com a única solução:
$$x_{\pi(i)} = b''_i \qquad \mbox{ para todo } 1\leq i \leq n.$$ 
\end{block}


\end{frame}

\begin{frame}{\bf 3.4}{Possível indeterminado}

\begin{block}{Sistemas possíveis indeterminados}
Se o sistema nem for impossível nem possível determinado então é {\bf possível indeterminado}. As soluções são da forma: 
$$ \mbox{ solução particular } + \mbox{ solução do sistema homogéneo}.$$
\pause A solução particlar é 

$$x_{\pi(i)} = b''_i \qquad \mbox{ para todo } 1\leq i \leq k \qquad x_{\pi(i)}=0 \qquad \mbox{ se } i>k.$$ 
\end{block}
\end{frame}

\begin{frame}{\bf 3.4}{Base do sistema homogéneo}
\begin{block}{}
Uma base do sistema homogéneo é dado pelos $n-k$ vectores $x^j$, $j=k+1, \ldots, n$:

$$x^j_{\pi(i)}=\left\{\begin{array}{cll} 
-a''_{ij} &\mbox{ se }& 1\leq i \leq k \\
1 & \mbox{ se } & i=j \\ 
0 & &\mbox{ nos restantes casos}\end{array}\right.$$
\end{block}
\end{frame}

\begin{frame}{Exemplo}
O sistema 
$$\left\{ \begin{array}{rcrcrcrcr}
x_1 & + & 2x_2  &+& 3x_3 &+& 4x_4 &=& 5 \\
2x_1 & + & 4x_2  &+& x_3 &+& 2x_4 &=& 4 \\
3x_1 & + & 6x_2  &+& 4x_3 &+& 6x_4 &=& 9 
\end{array}\right.$$
corresponde à matriz:

$$\left( \begin{array}{rrrr|r}
1 & 2& 3& 4& 5 \\
2&  4& 1& 2& 4 \\
3& 6& 4& 6& 9 
\end{array}\right)$$

\end{frame}


\begin{frame}{Exemplo}
No primeiro passo eliminamos as entradas da primeira coluna das linhas $L_2$ e $L_3$:

$$ L_2 \leftarrow L_2 - 2L_1 \qquad L_3 \leftarrow L_3 - 3L_1$$

$$\left( \begin{array}{rrrr|r}
1 & 2& 3& 4& 5 \\
2&  4& 1& 2& 4 \\
3& 6& 4& 6& 9 
\end{array}\right)
\rightarrow 
\left( \begin{array}{rrrr|r}
1 & 2& 3& 4& 5 \\
0&  0& -5& -6& -6 \\
0& 0& -5& -6& -6 
\end{array}\right)$$
É necessário trocar duas colunas, digamos coluna $C_2$ e coluna $C_4$.
\end{frame}


\begin{frame}{Exemplo}
$$\left( \begin{array}{rrrr|r}
1 & 2& 3& 4& 5 \\
0&  0& -5& -6& -6 \\
0& 0& -5& -6& -6 
\end{array}\right)
\leftrightarrow 
\left\{ \begin{array}{rrrrcr}
x_1 & +2x_2  &+3x_3 &+4x_4 &=&5 \\
    &        &-5x_3 &-6x_4 &=&-6 \\
    &        &-5x_3 &-6x_4 &=&-6 
\end{array}\right.$$
Trocar coluna $C_2$ e $C_4$.
$$\left( \begin{array}{rrrr|r}
1 & 4& 3& 2& 5 \\
0&  -6& -5& 0& -6 \\
0& -6& -5& 0& -6 
\end{array}\right)
\leftrightarrow 
\left\{ \begin{array}{rrrrcr}
x_1 &   +4x_4 &+3x_3 &+2x_2 &=&5 \\
    &   -6x_4      &-5x_3 & &=&-6 \\
    &    -6x_4     &-5x_3 &&=&-6 
\end{array}\right.$$
Logo a permutação é $\pi=(24)$.
\end{frame}


\begin{frame}{Exemplo}
A operação de linha $L_3 \leftarrow L_3+L_2$:
$$\left( \begin{array}{rrrr|r}
1 & 4& 3& 2& 5 \\
0&  -6& -5& 0& -6 \\
0& -6& -5& 0& -6 
\end{array}\right)
\rightarrow 
\left( \begin{array}{rrrr|r}
1 & 4& 3& 2& 5 \\
0&  -6& -5& 0& -6 \\
0& 0& 0& 0& 0 
\end{array}\right)$$
Logo o sistema é possível indeterminada.
\end{frame}

\begin{frame}{Exemplo}
Segunda redução: operação de linha $L_1 \leftarrow 6L_1+4L_2$:
$$\left( \begin{array}{rrrr|r}
1 & 4& 3& 2& 5 \\
0&  -6& -5& 0& -6 \\
0& 0& 0& 0& 0 
\end{array}\right)
\rightarrow 
\left( \begin{array}{rrrr|r}
6 & 0&  -2& 12& 6 \\
0&  -6& -5& 0& -6 \\
0& 0& 0& 0& 0 
\end{array}\right)$$
Divisão para normalizar os coeficientes da diagonal: $L_1 \leftarrow \frac{1}{6}L_1$ e $L_2 \leftarrow \frac{-1}{6}L_2$:
$$\left( \begin{array}{rrrr|r}
6 & 0&  -2& 12& 6 \\
0&  -6& -5& 0& -6 \\
0& 0& 0& 0& 0 
\end{array}\right)
\rightarrow 
\left( \begin{array}{rrrr|r}
1 & 0&  -\frac{1}{3}& 2& 1 \\
0&  1& \frac{5}{6}& 0& 1 \\
0& 0& 0& 0& 0 
\end{array}\right)$$
\end{frame}

\begin{frame}{Exemplo}{Solução particular}

Nota-se que 
$$\left( \begin{array}{rrrr|r}
1 & 0&  -\frac{1}{3}& 2& 1 \\
0&  1& \frac{5}{6}& 0& 1 \\
0& 0& 0& 0& 0 
\end{array}\right)
\leftrightarrow 
\left\{ \begin{array}{rrrrcr}
x_1 &   &-\frac{1}{3}x_3 &+2x_2 &=&1 \\
    &   x_4      &+\frac{5}{6}x_3 & &=&1 
\end{array}\right.
$$
As variáveis $x_3$ e $x_2$ são livres e uma solução particular obtemos se $x_3=x_2=0$, i.e. $x=(1,0,0,1)$ ou seja:
$$x_{\pi(1)}=x_1=1 \qquad x_{\pi(2)}=x_4=1 \qquad x_{\pi(3)}=x_3=0 \qquad x_{\pi(4)}=0$$

\end{frame}

\begin{frame}{Exemplo}{Solução do sistema homogeneo}

A base do sistema homogeneo é dado por $2$ vectores:
\pause

Caso $j=3$: $u= (\frac{1}{3}, 0, 1, -\frac{5}{6})$, porque
$$u_{\pi(1)}=u_1=\frac{1}{3} \qquad u_{\pi(2)}=u_4=-\frac{5}{6} \qquad u_{\pi(3)}=u_3=1, u_{\pi(4)}=u_2=0$$
\pause
Alternativamente podemos usar $6u=(2,0,6,-5)$.

\medskip\pause


Caso $j=4$: $v = (-2, 1, 0, 0)$, porque
$$v_{\pi(1)}=v_1=-2 \qquad v_{\pi(2)}=v_4=0 \qquad v_{\pi(3)}=v_3=0, v_{\pi(4)}=v_2=1$$


\end{frame}


\begin{frame}{Exemplo}{Solução geral}

Seja $S$ o conjunto das soluções do sistema.

\begin{eqnarray*}
S&=& \left \{(1,0,0,1) + \lambda (2,0,1,-5) + \mu (-2,1,0,0) \mid \lambda,\mu \in K\right\}\\
&=& \left \{  (1+2(\lambda-\mu), \mu, 6\lambda, 1-5\lambda) \mid \lambda,\mu \in K\right\}.
\end{eqnarray*}



\end{frame}

\begin{frame}[fragile]{\bf Algoritmo}{1º Passo}
\begin{algorithmic}
\REQUIRE matriz $A$: colunas $C_1, \ldots, C_n$; linhas $L_1, \ldots, L_m$ e $b \in K^m$
\ENSURE conjunto de soluções de $Ax=b$.
  \STATE $k \leftarrow 0$
  \STATE $\pi\leftarrow id$
  \FOR{$1\leq i \leq n$}
			\IF {coluna $i$ é zero a partir da linha $i$}
				\STATE trocar $C_i \leftrightarrow C_{n-k}$; $\pi \leftarrow (i, n-k)\pi $
				\STATE $k\leftarrow k+1$
			 \ELSE
			 	\STATE trocar linhas da matriz $(A|b)$ tal que $a_{ii}\neq 0$.
				\STATE eliminação das entradas por baixo de $a_{ii}$ por operações de linhas em $(A|b)$.
			\ENDIF
   \ENDFOR
\end{algorithmic}
\end{frame}

\begin{frame}[fragile]{\bf Algoritmo}{2º Passo}
\begin{algorithmic}
  \IF {$(b_{n-k+1}, \ldots, b_n)\neq (0,\ldots,0)$}
  	\RETURN $\emptyset$
  \ELSE
    \FOR{$n-k\geq i \geq 1$}
				\STATE eliminação das entradas por cima de $a_{ii}$ por operações de linhas em $(A|b)$.
   \ENDFOR
     \IF {$k=0$}
  	\RETURN $b$
    \ELSE 
    	  \STATE define $x^j$ for all $n-k<j\leq n$ por $x^j_{\pi(i)}=-a_{ij}$ se $i\leq n-k$ e $x^j_{\pi(i)} = \delta_{ij}$ se $i>n-k$.
	  \RETURN $(b, x^{n-k+1}, \ldots, x^n)$
    \ENDIF
  \ENDIF
\end{algorithmic}
\end{frame}

\end{document}





3. 
3.1 O template de vectores
3.2 O template de matrizes
3.3 Multiplicação rápida de matrizes: Algoritmo de Strassen.
3.4 Resolução de sistemas lineares

4. 
4.1 Introdução à geometria algébrica
4.2 Ordem nos monómios
4.3 Algoritmo da divisão para polinómios em múltiplos indeterminadas
4.4 Lemma de Dickson
4.5 Bases de Gröbner e Teorema de Hilbert
4.6 Algoritmo de Buchberger
4.7 Teoria de Eliminação
4.8 Teorema "Nullstellensatz" de Hilbert
4.9 Algoritmo de filiação de radicais
4.10 Aplicação à robótica
\end{frame}

