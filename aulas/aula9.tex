%\documentclass[handout]{beamer}
\documentclass{beamer}
\usepackage{algorithmic}
\usepackage{polynom}
\usepackage[english, portuguese]{babel}
\usepackage{hyperref}
\usepackage{amssymb}
\usepackage[utf8x]{inputenc}
\usepackage{euler}

\usepackage[center]{caption}

\usepackage{listings}
\lstset{language=C++, basicstyle=\footnotesize}

%\floatname{algorithm}{Função }
\renewcommand{\algorithmicrequire}{\textbf{Input:}}
\renewcommand{\algorithmicensure}{\textbf{Output:}}
\newcommand{\q}{\mathbf{q}}
\newcommand{\ZZ}{\mathbb Z}
\newcommand{\NN}{\mathbb N}
\newcommand{\CC}{\mathbb C}
\newcommand{\RR}{\mathbb R}
\newcommand{\QQ}{\mathbb Q}
\newcommand{\grau}[1]{\mathrm{grau}({#1})}
\newcommand{\mdc}[2]{\mathrm{mdc}({#1}, {#2})}
\newcommand{\mmc}[2]{\mathrm{mmc}({#1}, {#2})}
\newcommand{\quo}[2]{{#1}\, \mathrm{quo} \, {#2}}
\newcommand{\rem}[2]{{#1}\, \mathrm{rem} \, {#2}}

\newcommand{\cvector}[1]{\mathrm{vector}(#1)}
\newcommand{\lc}[1]{\mathrm{lc}(#1)}
%\usetheme{Berlin}
\usetheme[pageofpages=of,% String used between the current page and the
                         % total page count.
          bullet=circle,% Use circles instead of squares for bullets.
          titleline=true,% Show a line below the frame title.
          alternativetitlepage=true,% Use the fancy title page.
          %titlepagelogo=logotipo_pdf1,% Logo for the first page.
          %watermark=logotipo_pdf1,% Watermark used in every page.
          %watermarkheight=100px,% Height of the watermark.
          %watermarkheightmult=2,% The watermark image is 4 times bigger
                                % than watermarkheight.
          ]{Torino}

\title[M342]{M342 Álgebra Computacional}
\author{Christian Lomp}
\institute{FCUP}
\date{24 de Outubro de 2011}
\begin{document}

\begin{frame}
\titlepage
\end{frame}


%3. 
%3.1 O template de vectores
%3.2 O template de matrizes
%3.3 Multiplicação rápida de matrizes: Algoritmo de Strassen.
%3.4 Resolução de sistemas lineares





\begin{frame}{\bf 3.4 Resolução de sistemas lineares}


$$\left\{ \begin{array}{cccccccc}
a_{11}x_1 &+& a_{12}x_2 & \ldots &+& a_{1n}x_n &=& b_1 \\
a_{21}x_1 &+& a_{22}x_2 & \ldots &+& a_{2n}x_n &=& b_2 \\
   \vdots &&           & \ddots &&           & &\vdots \\
a_{m1}x_1 &+& a_{m2}x_2 & \ldots &+& a_{mn}x_n &=& b_m \\
\end{array}\right.$$
\end{frame}


\begin{frame}{\bf 3.4 Resolução de sistemas lineares }{Stratégia}
{\small 
$$
\left( \begin{array}{cccc|l}
a_{11} & a_{12} & \ldots & a_{1n} & b_1 \\
a_{21} & a_{22} & \ldots & a_{2n} & b_2 \\
   \vdots &     & \ddots &      &\vdots \\
a_{m1} & a_{m2} & \ldots & a_{mn} & b_m \\
\end{array}\right)\rightarrow
\left( \begin{array}{cccccc|l}
a'_{11}   & a'_{12} & \ldots  & \ldots  & \ldots  & a'_{1n}  & b'_1 \\
0         & a'_{22} & \ldots  &   &   & a'_{2n}  & b'_2 \\
   \vdots & \ddots  & \ddots  &   &   &  \vdots        & \vdots \\
 0        & \ldots  &  0      & a'_{kk} & \ldots  & a'_{kn}  & b'_k \\
\vdots   &         & \vdots  & 0       & \ldots  & 0        & b'_{k+1} \\
         &         &         &  \vdots       &   &   \vdots & \vdots \\
0         & \ldots  & 0      &  0       &  \ldots       & 0         & b'_m \\
\end{array}\right)
$$}

\end{frame}

\begin{frame}{\bf 3.4}{}
$$\left\{ \begin{array}{lcl} \mbox{ impossível } & \mbox{ se } & (b_{k+1}, \ldots, b_m) \neq (0,\ldots, 0)\\ 
 \mbox{ possível determinado } & \mbox{ se } & (b_{k+1}, \ldots, b_m) = (0,\ldots, 0) \\
&& \left(\begin{array}{ccc} a'_{1{k+1}} & \ldots & a'_{1n} \\ \vdots & &\vdots \\ a'_{k{k+1}} & \ldots & a'_{kn} \end{array}\right)
= \left(\begin{array}{ccc} 0 & \ldots & 0 \\ \vdots & &\vdots \\ 0 & \ldots & 0 \end{array}\right)\\
 \mbox{ possível indeterminado } & \mbox{ } & \mbox{ nos restantes casos}
\end{array}\right.$$
\end{frame}


\end{document}





3. 
3.1 O template de vectores
3.2 O template de matrizes
3.3 Multiplicação rápida de matrizes: Algoritmo de Strassen.
3.4 Resolução de sistemas lineares

4. 
4.1 Introdução à geometria algébrica
4.2 Ordem nos monómios
4.3 Algoritmo da divisão para polinómios em múltiplos indeterminadas
4.4 Lemma de Dickson
4.5 Bases de Gröbner e Teorema de Hilbert
4.6 Algoritmo de Buchberger
4.7 Teoria de Eliminação
4.8 Teorema "Nullstellensatz" de Hilbert
4.9 Algoritmo de filiação de radicais
4.10 Aplicação à robótica
\end{frame}

