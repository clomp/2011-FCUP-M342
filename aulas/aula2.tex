\documentclass{beamer}
\usepackage{algorithmic}
\usepackage{polynom}
\usepackage[english, portuguese]{babel}
\usepackage{hyperref}
\usepackage{amssymb}
\usepackage[utf8x]{inputenc}
\usepackage{euler}
\usepackage{listings}
\lstset{language=C++, basicstyle=\footnotesize}

%\floatname{algorithm}{Função }
\renewcommand{\algorithmicrequire}{\textbf{Input:}}
\renewcommand{\algorithmicensure}{\textbf{Output:}}
\newcommand{\q}{\mathbf{q}}
\newcommand{\ZZ}{\mathbb Z}
\newcommand{\cvector}[1]{\mathrm{vector}(#1)}
%\usetheme{Berlin}
\usetheme[pageofpages=of,% String used between the current page and the
                         % total page count.
          bullet=circle,% Use circles instead of squares for bullets.
          titleline=true,% Show a line below the frame title.
          alternativetitlepage=true,% Use the fancy title page.
          %titlepagelogo=logotipo_pdf1,% Logo for the first page.
          %watermark=logotipo_pdf1,% Watermark used in every page.
          %watermarkheight=100px,% Height of the watermark.
          %watermarkheightmult=2,% The watermark image is 4 times bigger
                                % than watermarkheight.
          ]{Torino}

\title[M342]{M342 Álgebra Computacional}
\author{Christian Lomp}
\institute{FCUP}
\date{14 de setembro de 2011}
\begin{document}

\begin{frame}
\titlepage
\end{frame}





\section{2. Estruturas de dados}

\begin{frame}{\bf 2. Estruturas de dados}{\bf 2.1 Representação de número nos computador e limitações}

\begin{block}{Representação $\q$-ária}
Seja $\q$ um número positivo. Qualquer número positvo $x\in \ZZ$ tem uma representação única:
$$ x = a_0 + a_1\q + a_2\q^2 + a_3\q^3 + \cdots a_n\q^n$$
onde $a_i \in \{0,1,\ldots, \q-1\}$.
\end{block}

Representamos um $x$ pelo vector dos coeficientes de representação $\q$-ária, i.e.
$$x = {(a_na_{n-1}\cdots a_1)}_{\q}$$

\end{frame}



\begin{frame}[fragile]
\begin{algorithmic}
\REQUIRE inteiros positivos $x$, $\q$.
\ENSURE $v=(a_n \ldots a_0)$ tal que $x=\sum_{i=0}^n a_n \q^n$.
	\STATE $n\leftarrow 0$
	\WHILE{$x>0$}
  		\STATE $a_n \leftarrow x\%\q$ \COMMENT{resto da divisão por $\q$}
  		\STATE $x \leftarrow x/\q$ \COMMENT{divisão por $\q$}
  		\STATE $n\leftarrow n+1$
	\ENDWHILE
\end{algorithmic}
\end{frame}

\begin{frame}
 \begin{block}{Exemplo}
  \begin{itemize}
   \item $(342)_{10} = 1\cdot 2^8 + 1\cdot 2^6 + 1\cdot 2^4+1\cdot 2^2 + 1\cdot 2^1 = (101010110)_2$
   \item $(342)_{10} = 1\cdot 3^5 + 1\cdot 3^4 + 2\cdot 3^2 = (110200)_3$
   \item $(342)_{10} = 1\cdot {16}^2 + 5\cdot {16}^1 + 6\cdot {16}^0 =  (156)_{16}$   
   \item $(342)_{10} = 5\cdot 8^2 + 2\cdot {8}^1 + 6\cdot {8}^0 =  (526)_{8}$   
   \item $(342)_{8} = 3\cdot {8}^2 + 4\cdot {8}^1 + 2\cdot {8}^0 =  (226)_{10} = (1401)_5 = (E 2)_{16}$
  \end{itemize}

 \end{block}

\end{frame}



\begin{frame}{\bf 2.}{\bf 2.1 Representação de número nos computador e limitações}
O tipo {\it unsigned int} pode representar números entre $0$ e $2^{32}-1$. Seja $\q=2^{32}$.

\begin{block}{Representação $2^{32}$-ária}
Identificamos um tuple $x=(s,v)$ onde $s\in \{0,1\}$ $v=\cvector{a_0,a_1,\ldots, a_n}$ e $0\leq a_i < 2^{32}$ com o número 
$$ (-1)^s a_0 + a_12^{32} + a_22^{64} + a_32^{96} + \cdots a_n2^{32n}$$
\end{block}


\end{frame}



\begin{frame}[fragile]{\bf 2.}{\bf 2.1 Representação de inteiros no computador}

\begin{lstlisting}[language=C++]
#include <vector>

class inteiros {

  int s;
  vector<int> v;

  inteiros (int sign, vector<int> coeficents)
  {
    s = sign;
    v = coeficents;
  }
}

\end{lstlisting}
\end{frame}


\begin{frame}{Classes}

\begin{enumerate} 
 \item \pause Pode-se dizer que {\it Classes} são estruturas de dados com funções pre-definidos.
\item \pause Uma classe contem um conjunto de variáveis (atributos) e certas funções (métodos).
\item \pause Os atriubutos  e métodos podem ser privados ({\it private}) ou públicos ({\it public}).
\end{enumerate}

\end{frame}

\begin{frame}[fragile]{STL vector}
A classe {\it vector} é uma classe padrão de $C++$ e permite guardar uma lista de objectos (no caso de {\it vector $<$int$>$} é uma lista de inteiros).

Alguns funções que {\it vector} oferece são:
\begin{enumerate}
 \item {\it push\_back} : adicionar um elemento no fim da lista
\item {\it size} : tamanho da lista
\item {\it insert}: inserir elemento numa posição
\end{enumerate}

\begin{lstlisting}[language=C++]
 vector<int> lista;
 lista.push_back(2);
 lista.push_back(3);
 cout << "No.Elementos:" <<  lista.size() << endl;
 cout << "O primeiro elemento: " << lista[0] << endl;
\end{lstlisting}

\end{frame}


\begin{frame}{\bf 2.}{\bf Adição de inteiros}

Sejam $x=\sum_{i=0}^n a_i \q^i$ e $y=\sum_{j=0}^n b_j \q^j$. Então 
$$x+y = \sum_{i=0}^n (a_i+b_i) \q^i.$$
Como $0\leq a_i, b_i \leq \q-1$, $ 0\leq a_i + b_i \leq 2\q-2.$

Seja $\gamma_{-1} = 0$ e para todo $i\geq 0$ define
{\small $$\gamma_i = \left\{ \begin{array}{cc} 0 & \mbox{ se } a_i+b_i+\gamma_{i-1} < \q\\ 1 & \mbox{ se } a_i+b_i+\gamma_{i-1} \geq \q \end{array} \right. \qquad
 c_i = \left\{ \begin{array}{lc} a_i+b_i+\gamma_{i-1} & \mbox{ se } \gamma_i=0 \\ a_i+b_i+\gamma_{i-1}-\q &  \mbox{ se } \gamma_i=1 \end{array} \right. $$}
Portanto $$x+y = \sum_{i=0}^{n+1} c_i \q^i.$$
 
\end{frame}




\begin{frame}[fragile]{\bf 2.}{\bf Adição de inteiros}
\begin{algorithmic}
\REQUIRE inteiros $x=(s,(a_0,\ldots, a_n))$, $y=(s,(b_0,\ldots, b_n))$
\ENSURE inteiro $z=(s, (c_0, \ldots, c_{n+1}))$ tal que $z=x+y$.
\STATE $\gamma \leftarrow 0$ 
\FOR{ $i=0,\ldots, n$ }
  \STATE $c_i \leftarrow a_i+b_i+\gamma$
  \IF {$c_i\geq 2^{32}$}
        \STATE $c_i \leftarrow c_i - 2^{32}$
        \STATE $\gamma \leftarrow 1$
  \ELSE
	\STATE $\gamma \leftarrow 0$
  \ENDIF
\ENDFOR
\STATE $c_{n+1}\leftarrow \gamma$
\end{algorithmic}

\end{frame}


\begin{frame}{Exemplo}
\begin{tabular}{lcl}
$x=(0,(2^{32}-1, 1))$ &;    & {\tiny $2^{32}-1 + 2^{32} = 858993458$}\\
$y=(0, (2^{32}-2, 2^{32}-1))$ &;  &  {\tiny $2^{32}-2 + (2^{32}-1)*2^{32} = 18446744073709551614$.}
\end{tabular}
\begin{center}
\begin{tabular}{|r|r|c|l|}\hline
 $i$ & $c_i$ & $\gamma$ & \\\hline
$0$ & $2^{32}-3$ & $\gamma \leftarrow 1$ & pois $2^{32}-1 + 2^{32}-2 \geq 2^{32}  $\\\hline
$1$ & $1$ & $\gamma \leftarrow 1$ & pois $1+2^{32}-1 + 1 \geq 2^{32}  $\\\hline
$2$ & $1$ && \\\hline
\end{tabular}
\end{center}

Logo $x+y= (0, (2^{32}-3, 1,1))$ e 
$$ 2^{32}-3 + 2^{32} + 2^{64} = (2^{32}-1 + 2^{32} ) + (2^{64} - 2) = 18446744074568544072.$$

\end{frame}



\begin{frame}[fragile]{\bf 2.}{\bf Comparar inteiros}

\begin{block}{Igualdade}
 Dois inteiros $x=(s,(a_0,\ldots, a_n))$ e $y=(s,(b_0,\ldots, b_m))$ são iguais se
$$s=t \qquad n=m \qquad a_i=b_i \:\:\forall 0\leq i\leq n.$$
\end{block}
 
\end{frame}


\begin{frame}[fragile]{\bf 2.}{\bf Comparar inteiros}

\begin{block}{Ordem}
 O inteiro $x=(s,(a_0,\ldots, a_n))$ e menor do que $y=(s,(b_0,\ldots, b_m))$ se e só se
\begin{itemize}
 \item[] $s<t$ ou
 \item[] $(s=t) \wedge (n<m)$  ou 
 \item[] $(s=t) \wedge (n=m) \wedge \exists 0\leq i \leq n: a_i<b_i \wedge \forall i<j\leq n: a_j=b_j$.
\end{itemize}

\end{block}

\end{frame}


\begin{frame}[fragile]{\bf 2.}{\bf Subtracção de inteiros}
\begin{algorithmic}
\REQUIRE inteiros $x=(s,(a_0,\ldots, a_n))$, $y=(s,(b_0,\ldots, b_n))$ tal que $y<x$.
\ENSURE inteiro $z=(s, (c_0, \ldots, c_{n+1}))$ tal que $z=x-y$.
\STATE $\gamma \leftarrow 0$ 
\FOR{ $i=0,\ldots, n$ }
  \STATE $c_i \leftarrow a_i-b_i-\gamma$
  \IF {$c_i\geq 0$}
        \STATE $c_i \leftarrow c_i + 2^{32}$
        \STATE $\gamma \leftarrow 1$
  \ELSE
	\STATE $\gamma \leftarrow 0$
  \ENDIF
\ENDFOR
\end{algorithmic}

\end{frame}




\begin{frame}{\bf 2.}{\bf Multiplicação de inteiros}
\begin{block}{Multiplicação} Sejam $\q=2^{32}$, $x=\sum_{i=0}^n a_i\q^i$ e $y=\sum_{j=0}^m b_j\q^j$. Então
$$ xy = \sum_{i=0}^n \sum_{j=0}^m a_ib_j \q^{i+j}.$$
Como $0\leq a_i, b_j < \q$, tem-se que $a_ib_j = c_{i,j} + d_{i,j}\q$ onde $0\leq c_{i,j}, d_{i,j} < \q$.
$$ xy = \sum_{i=0}^n \sum_{j=0}^m c_{i,j} \q^{i+j} + d_{i,j}\q^{i+j+1}.$$
\end{block}
\end{frame}


\begin{frame}[fragile]{\bf 2.}{\bf Multiplicação de inteiros}
Suponha que existe uma função {\bf mult}({\it int} a,{\it int} b) cujo resultado é um par $(c,d)$ de {\bf int} tal que 
$$ab = c + 2^{32}d$$

\begin{algorithmic}
\REQUIRE inteiros $x=(s,(a_0,\ldots, a_n))$, $y=(t,(b_0,\ldots, b_m))$.
\ENSURE inteiro $z=(u, (c_0, \ldots, c_{nm}))$ tal que $z=xy$.
\STATE $z \leftarrow ( (s+t)\mathrm{mod} 2, (0))$ 
\FOR{ $i=0,\ldots, n$ }
 \FOR{ $j=0,\ldots, m$ }
  \STATE $(c,d)\leftarrow {\mathbf{mult}}(a_i,b_j)$
  \STATE $z \leftarrow z + c 2^{(i+j)32}$
  \STATE $z \leftarrow z + d 2^{(i+j+1)32}$
 \ENDFOR
\ENDFOR
\end{algorithmic}

\end{frame}
\end{document}



\begin{frame}{\bf 2.}{\bf Divisão com resto}
 
Como dividir inteiros ?
\[
\begin{array}{rlcl}
&6\q^3 + 3\q^2 + 4 & : & \underline{2\q^2 + \q + 1} \\
-&\underline{(6\q^3 + 3\q^2 + 3\q)} && 3\q \\
&-3\q +4  &&  \\
\end{array} \]
Portanto 
$$6\q^3 + 3\q^2 + 4  = 3\q (2\q^2+\q +1 ) + (-3\q + 4)$$
\end{frame}





\begin{frame}

2.2 A classes dos números racionais
2.3 A classes dos inteiros modulares
2.4 O template dos polinómios.
2.5 Multiplicação de polinómios: Algoritmo de Karatsuba

3. 
3.1 O template de vectores
3.2 O template de matrizes
3.3 Multiplicação rápida de matrizes: Algoritmo de Strassen.
3.4 Resolução de sistemas lineares

4. 
4.1 Introdução à geometria algébrica
4.2 Ordem nos monómios
4.3 Algoritmo da divisão para polinómios em múltiplos indeterminadas
4.4 Lemma de Dickson
4.5 Bases de Gröbner e Teorema de Hilbert
4.6 Algoritmo de Buchberger
4.7 Teoria de Eliminação
4.8 Teorema "Nullstellensatz" de Hilbert
4.9 Algoritmo de filiação de radicais
4.10 Aplicação à robótica
\end{frame}

