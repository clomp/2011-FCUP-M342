\documentclass[handout]{beamer}
\usepackage{algorithmic}
\usepackage{polynom}
\usepackage[english, portuguese]{babel}
\usepackage{hyperref}
\usepackage{amssymb}
\usepackage[utf8x]{inputenc}
\usepackage{euler}
\usepackage{listings}
\lstset{language=C++, basicstyle=\footnotesize}

%\floatname{algorithm}{Função }
\renewcommand{\algorithmicrequire}{\textbf{Input:}}
\renewcommand{\algorithmicensure}{\textbf{Output:}}
\newcommand{\q}{\mathbf{q}}
\newcommand{\ZZ}{\mathbb Z}
\newcommand{\grau}[1]{\mathrm{grau}({#1})}
\newcommand{\cvector}[1]{\mathrm{vector}(#1)}
\newcommand{\lc}[1]{\mathrm{lc}(#1)}
%\usetheme{Berlin}
\usetheme[pageofpages=of,% String used between the current page and the
                         % total page count.
          bullet=circle,% Use circles instead of squares for bullets.
          titleline=true,% Show a line below the frame title.
          alternativetitlepage=true,% Use the fancy title page.
          %titlepagelogo=logotipo_pdf1,% Logo for the first page.
          %watermark=logotipo_pdf1,% Watermark used in every page.
          %watermarkheight=100px,% Height of the watermark.
          %watermarkheightmult=2,% The watermark image is 4 times bigger
                                % than watermarkheight.
          ]{Torino}

\title[M342]{M342 Álgebra Computacional}
\author{Christian Lomp}
\institute{FCUP}
\date{19 de setembro de 2011}
\begin{document}

\begin{frame}
\titlepage
\end{frame}





\section{2. Estruturas de dados}

\begin{frame}[fragile]{\bf 2. Estruturas de dados}{\bf A classe inteiro}

\lstset{language=C++,basicstyle=\tiny}
\begin{lstlisting}[language=C++]
#include <vector>
#include <inteiro>

class inteiro {

  bool sinal;
  unsigned int base;
  vector<unsigned int> coeficientes;

public:
  inteiro();
  void operator = (inteiro); // copiar 
  bool operator < (inteiro); // comparar
  bool operator == (inteiro); // comparar
  inteiro operator + (inteiro); // somar
  inteiro operator - (inteiro); // subtrair
  inteiro operator * (inteiro); // multiplicar
  inteiro operator / (inteiro); // quociente
  inteiro operator % (inteiro);  // resto
}

\end{lstlisting}
\end{frame}


\begin{frame}{\bf 2.}{\bf Aritmética dos polinómios}
\begin{block}{O anel dos polinómios}
Um polinómio não nulo $f=\sum_{i=0}^n a_i x^i \in R[x]$ com coeficientes num anel comutativo $R$ é únicamente determinado pelos seus coeficientes $(a_0,a_1,\ldots, a_n)$ onde $a_n\neq 0$. \pause Neste caso chama-se $n$ o {\it grau} do polinómio. $\grau{f}=n$. No caso do polinómio nulo $0$ escrevemos $\grau{0}=-\infty$.
\pause
O coeficiente $a_n$ chama-se o {\bf coeficiente guia} ({\it leading coefficient}) denotado por $a_n=\lc{f}$.
\end{block}
\end{frame}

\begin{frame}{\bf 2.}
Representamos polinómios pela sucesão dos seus coeficientes.

\pause 
Dois polinómios não-nulos $f=(a_0,\ldots, a_n)$ e $g=(b_0,\ldots, b_m))$ são iguais se
$$n=m \qquad a_i=b_i \:\:\forall 0\leq i\leq n.$$

\end{frame}



\begin{frame}[fragile]{\bf 2.}{\bf Representação de polinómios com C++}

\begin{lstlisting}[language=C++]
#include <vector>
#include <inteiro>

class polinomio {

  bool nulo;
  vector<inteiro> coeficientes;

public:
  polinomio();
  unsigned int grau();
  inteiro leadingCoeficient();
  polinomio operator + (polinomio);
  polinomio operator - (polinomio);
  polinomio operator * (polinomio);
  ...
}

\end{lstlisting}
\end{frame}


\begin{frame}{\bf 2.}{\bf Adição de Polinómios}

Sejam $f=\sum_{i=0}^n a_i x^i$ e $g=\sum_{j=0}^m b_j x^j$ com $n\leq m$. então 
$$f+g = \sum_{i=0}^n (a_i+b_i) x^i  + \sum_{i=n+1}^m b_i x^i .$$
\end{frame}




\begin{frame}[fragile]{\bf 2.}{\bf Adição/Subtração de Polinómios}
\begin{algorithmic}
\REQUIRE polinómios $f=(a_0,\ldots, a_n)$, $g=(b_0,\ldots, b_m)$ com $n\leq m$
\ENSURE polinómio $h=(c_0, \ldots, c_m)$ tal que $h=f\pm g$.
\FOR{ $i=0,\ldots, n$ }
  \STATE $c_i \leftarrow a_i\pm b_i$
\ENDFOR
\FOR{ $i=n+1,\ldots, m$ }
  \STATE $c_i \leftarrow b_i$
\ENDFOR
\end{algorithmic}

\end{frame}




\begin{frame}{\bf 2.}{\bf Multiplicação de polinómios}
%\begin{block}{Multiplicação} 
Sejam $f=(a_0,\ldots, a_n)$ e $g=(b_0,\ldots, b_m))$ dois polinómios não-nulos.
Então
$$ fg = (c_0, \ldots, c_{nm}) \qquad \mbox{ com } c_k=\sum_{i+j=k}a_i b_j .$$
\pause Alternativamente 
$$ fg = \sum_{i=0}^n a_ix^ig.$$
onde $a_ix^i$ é optido por uma translação e multiplicação escalar: 
$$(b_0,\ldots, b_m) \mapsto (\underbrace{0,\ldots, 0}_{i-\mbox{posições}},a_ib_0,\ldots, a_ib_m).$$
%\end{block}
\end{frame}


\begin{frame}[fragile]{\bf 2.}{\bf Multiplicação de polinómios}
\begin{algorithmic}
\REQUIRE polinómios $f=(a_0,\ldots, a_n)$, $g=(b_0,\ldots, b_m)$.
\ENSURE polinómio $h=(c_0, \ldots, c_{nm})$ tal que $h=fg$.
\STATE $h \leftarrow (0)$ 
\FOR{ $i=0,\ldots, n$ }
  \STATE $z \leftarrow shift(i,g)$
  \STATE $h \leftarrow h + a_i*z$;
\ENDFOR
\RETURN h
\end{algorithmic}

\end{frame}

\begin{frame}[fragile]{\bf 2.}{\bf Multiplicação por um monómio}
\begin{algorithmic}
\REQUIRE polinómio $g=(b_0,\ldots, b_m)$ e exponente $i$
\ENSURE polinómio $h=shift(i,g) = x^ig$
\STATE $h \leftarrow (\underbrace{0,\ldots, 0}_{i-\mbox{posições}})$ 
\FOR{ $j=0,\ldots, m$ }
  \STATE $h_{i+j} \leftarrow b_j$
\ENDFOR
\RETURN h
\end{algorithmic}

\end{frame}

\begin{frame}[fragile]{\bf 2.}{\bf Multiplicação por um escalar}
\begin{algorithmic}
\REQUIRE polinómio $g=(b_0,\ldots, b_m)$ e escalar $a$
\ENSURE polinómio $h=a*g$
\FOR{ $j=0,\ldots, m$ }
  \STATE $h_j \leftarrow a*b_j$
\ENDFOR
\RETURN h
\end{algorithmic}

\end{frame}



\begin{frame}{\bf 2.}{\bf Divisão com resto}
\begin{block}{}
Dados inteiros $a$ e $b$ existem inteiros $q$ e $r$ tais que 
$$ a = qb + r, \qquad \mbox{com } |r|<|b|.$$
\pause Dado um anel comutativo $R$ e polinómios $f,g\in R[x]$ com $g\neq 0$ queremos encontrar polinómios $q$ e $r$ tais que 
$$ f=qg+r \qquad \mbox{com } \grau{r}<\grau{g}.$$
\end{block}
\pause
Nem sempre isto é possível: $R=\ZZ$, $f=x^2$ e $g=2x+1$.
\end{frame}


\begin{frame}
$$\polylongdiv[stage=10,style=A]{3x^4+2x^3+x+5}{x^2+2x+3}$$
\end{frame}

\begin{frame}[fragile]{\bf 2.}{\bf Divisão de polinómios}
\begin{algorithmic}
\REQUIRE polinómios $f=(a_0,\ldots, a_n)$, $g=(b_0,\ldots, b_m)$ com $m\leq n$ e $b_m$ invertível.
\ENSURE polinómios $q=(c_0, \ldots, c_k)$ e $r=(d_0, \ldots, d_l)$ tal que $f=qg+r$ e $\grau{r}<\grau{g}$.
	\STATE $r\leftarrow f$
	\STATE $q\leftarrow (0)$
	\FOR {$i=n-m, n-m-1, \ldots , 0$}
		\IF{$\grau{r}=m+i$}
			\STATE $q\leftarrow q+\lc{r}b_m^{-1}x^i$
			\STATE $r\leftarrow r-\lc{r}b_m^{-1}x^i g$
		\ENDIF
	\ENDFOR
	\RETURN $(q,r)$
\end{algorithmic}

\end{frame}

%%Programing example of division and of rest in C++


\end{document}





\begin{frame}

2.2 A classes dos números racionais
2.3 A classes dos inteiros modulares
2.4 O template dos polinómios.
2.5 Multiplicação de polinómios: Algoritmo de Karatsuba

3. 
3.1 O template de vectores
3.2 O template de matrizes
3.3 Multiplicação rápida de matrizes: Algoritmo de Strassen.
3.4 Resolução de sistemas lineares

4. 
4.1 Introdução à geometria algébrica
4.2 Ordem nos monómios
4.3 Algoritmo da divisão para polinómios em múltiplos indeterminadas
4.4 Lemma de Dickson
4.5 Bases de Gröbner e Teorema de Hilbert
4.6 Algoritmo de Buchberger
4.7 Teoria de Eliminação
4.8 Teorema "Nullstellensatz" de Hilbert
4.9 Algoritmo de filiação de radicais
4.10 Aplicação à robótica
\end{frame}

