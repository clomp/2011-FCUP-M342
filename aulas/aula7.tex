%\documentclass[handout]{beamer}
\documentclass{beamer}
\usepackage{algorithmic}
\usepackage{polynom}
\usepackage[english, portuguese]{babel}
\usepackage{hyperref}
\usepackage{amssymb}
\usepackage[utf8x]{inputenc}
\usepackage{euler}

\usepackage[center]{caption}

\usepackage{listings}
\lstset{language=C++, basicstyle=\footnotesize}

%\floatname{algorithm}{Função }
\renewcommand{\algorithmicrequire}{\textbf{Input:}}
\renewcommand{\algorithmicensure}{\textbf{Output:}}
\newcommand{\q}{\mathbf{q}}
\newcommand{\ZZ}{\mathbb Z}
\newcommand{\NN}{\mathbb N}
\newcommand{\CC}{\mathbb C}
\newcommand{\RR}{\mathbb R}
\newcommand{\QQ}{\mathbb Q}
\newcommand{\grau}[1]{\mathrm{grau}({#1})}
\newcommand{\mdc}[2]{\mathrm{mdc}({#1}, {#2})}
\newcommand{\mmc}[2]{\mathrm{mmc}({#1}, {#2})}
\newcommand{\quo}[2]{{#1}\, \mathrm{quo} \, {#2}}
\newcommand{\rem}[2]{{#1}\, \mathrm{rem} \, {#2}}

\newcommand{\cvector}[1]{\mathrm{vector}(#1)}
\newcommand{\lc}[1]{\mathrm{lc}(#1)}
%\usetheme{Berlin}
\usetheme[pageofpages=of,% String used between the current page and the
                         % total page count.
          bullet=circle,% Use circles instead of squares for bullets.
          titleline=true,% Show a line below the frame title.
          alternativetitlepage=true,% Use the fancy title page.
          %titlepagelogo=logotipo_pdf1,% Logo for the first page.
          %watermark=logotipo_pdf1,% Watermark used in every page.
          %watermarkheight=100px,% Height of the watermark.
          %watermarkheightmult=2,% The watermark image is 4 times bigger
                                % than watermarkheight.
          ]{Torino}

\title[M342]{M342 Álgebra Computacional}
\author{Christian Lomp}
\institute{FCUP}
\date{12 de Outubro de 2011}
\begin{document}

\begin{frame}
\titlepage
\end{frame}


%3. 
%3.1 O template de vectores
%3.2 O template de matrizes
%3.3 Multiplicação rápida de matrizes: Algoritmo de Strassen.
%3.4 Resolução de sistemas lineares





\begin{frame}{\bf 3. Álgebra linear }
Neste capítulo falamos de aspectos computacionais de álgebra linear: matrizes, vectores, sistemas de equações lineares
\end{frame}

\begin{frame}[fragile]{\bf 3.1 Vectores}
Para trabalhar com vectores definimos uma classe VETOR.

\lstset{language=C++}
\begin{lstlisting}
class VETOR
{
   vector<Tipo> data;
 public:
   VETOR(vector<Tipo> d) { data=d; };    // construtor
  ...
}
\end{lstlisting}

Aqui {\it Tipo} podia ser substiuido por um típo como {\it int}, {\it float} ou classes que definímos como {\it inteiro}, {\it racional}, {\it polinomio}, {\it modular}.

\end{frame}

\begin{frame}[fragile]{\bf 3.1 Vectores}{\bf Soma}
Se {\it Tipo} tem uma operação $+$ definida podemos definir a soma de vectores:
para $v=(v_1,\ldots, v_n)$ e $w=(w_1,\ldots, w_n)$ define-se a soma por
$$ v + w = (v_1+w_1, v_2+w_2, \ldots, v_n+w_n).$$
\lstset{language=C++}
\begin{lstlisting}
   VETOR VETOR::operator + (VETOR b)
   {
     vector soma<Tipo>;                 
     for (int i=0; i<data.size(); i++)  
       soma.push_back(data[i]+b.data[i]);
     return VETOR(soma);
   };
\end{lstlisting}
\end{frame}

\begin{frame}[fragile]{\bf 3.1 Vectores}{\bf Multiplicação escalar}
Se {\it Tipo} tem uma operação $*$ definida podemos definir a multiplicação escalar:
Para $v=(v_1,\ldots, v_n)$ e um escalar $\lambda$ define-se 
$$ \lambda v = (\lambda v_1, \lambda +v_2,  \ldots, \lambda v_n).$$
\lstset{language=C++}
\begin{lstlisting}
   VETOR VETOR::escalar(Tipo lambda) 
   {
     vector resultado<Tipo>;            
     for (int i=0; i<data.size(); i++)
       resultado.push_back(lambda*data[i]);
     return VETOR(resultado);
   };
\end{lstlisting}
\end{frame}

\begin{frame}[fragile]{\bf 3.1 Vectores}{\bf Diferença}
Se {\it Tipo} contém o elemento $-1$ podemos definir a diferença de vectores por $v-w := v + (-1)w$.

\lstset{language=C++}
\begin{lstlisting}
   VETOR VETOR::operator - (VETOR b)
   {
     return (*this)+b.escalar(-1);
   };
\end{lstlisting}
\end{frame}

\begin{frame}[fragile]{\bf 3.1 Vectores}{\bf Produto interno}
Para dois vectores $v=(v_1,\ldots, v_n)$ e $w=(w_1,\ldots, w_n)$ define-se o produto interno por
$$ v * w = v_1w_1 + v_2w_2 + \ldots v_nw_n.$$
\lstset{language=C++}
\begin{lstlisting}
   Tipo VETOR::operator * (VETOR b)
   {
     Tipo produto=0;              
     for (int i=0; i<data.size(); i++)
       produto = produto + (data[i]*b.data[i]);
     return produto;
   };
\end{lstlisting}
\end{frame}


\begin{frame}[fragile]{\bf 3.1 Vectores}{\bf Quadrado da norma}
O quadrado da norma de um vector $v$ é $|v|^2 = v*v$. 
\lstset{language=C++}
\begin{lstlisting}
   Tipo VETOR::Norma()
   {
     return (*this)*(*this);
   };
\end{lstlisting}
\end{frame}


%%%%%%%%%%%%%%%%%
 
\begin{frame}[fragile]{\bf 3.2 Matrizes}
Para trabalhar com matrizes definimos uma classe MATRIZ que consiste de um vector$<$VETOR$>$ onde fixamos agora o tipo {\it Tipo } nos componentes em VETOR.

\lstset{language=C++}
\begin{lstlisting}
class MATRIZ
{
   vector<VETOR> data;
 public:
   MATRIZ(vector<VETOR> A) { data=A; };    
  ...
}
\end{lstlisting}
Consideramos os componentes do vector {\it data} por linhas da matriz.
\end{frame}


\begin{frame}[fragile]{\bf 3.2 Matrizes}{\bf Soma}
A soma de duas matrizes $A=(a_{ij})$ e $B=(b_{ij})$ é 
$$ A + B = ( a_{ij}+b_{ij} ).$$
\lstset{language=C++}
\begin{lstlisting}
   MATRIZ MATRIZ::operator + (MATRIZ b)
   {
     vector<VETOR> soma;                 

     for (int i=0; i<data.size(); i++) 
      
       soma.push_back(data[i]+b.data[i]);
     
     return MATRIZ(soma);
   };
\end{lstlisting}
\end{frame}

\begin{frame}[fragile]{\bf 3.2 Matrizes}{\bf Multiplicação escalar}
Para $A=(a_{ij})$ e $\lambda$ define-se 
$$ \lambda A = (\lambda a_{ij}).$$
\lstset{language=C++}
\begin{lstlisting}
   MATRIZ MATRIZ::escalar(Tipo lambda) 
   {
     vector<VETOR> resultado;            

     for (int i=0; i<data.size(); i++)

       resultado.push_back(data[i].escalar(lambda));

     return MATRIZ(resultado);
   };
\end{lstlisting}
\end{frame}

\begin{frame}[fragile]{\bf 3.2 Matrizes}{\bf Diferença}

\lstset{language=C++}
\begin{lstlisting}
   MATRIZ MATRIZ::operator - (MATRIZ b)
   {
     return (*this)+b.escalar(-1);
   };
\end{lstlisting}
\end{frame}


\begin{frame}[fragile]{\bf 3.2 Matrizes}{\bf Colunas}
A $j$-esima coluna da matriz $A=(a_{ij})$ é o vetor $(a_{1j}, a_{2j}, \ldots, a_{nj})$.

\lstset{language=C++}
\begin{lstlisting}
   VETOR MATRIZ::Coluna(int j)
   {
     vector<Tipo> col;
     for (int i=0; i<data.size(); i++)
       col.push_back(data[i][j]);
     return VETOR(col);
   };
\end{lstlisting}
\end{frame}


\begin{frame}[fragile]{\bf 3.2 Matrizes}{\bf Multiplicação}
Para $A=(a_{ij})$ e $B=(b_{ij})$ define-se
$$ A*B = ( \sum_{k=0}^n a_{ik}b_{kj}).$$
\lstset{language=C++}
\begin{lstlisting}
   MATRIZ MATRIZ::operator * (MATRIZ b)
   {
     MATRIZ produto(data);
     int n=data.size();
     for(int i=0; i<n; i++)
       for(int j=0; j<n; j++)
         produto.data[i][j]=data[i]*b.Coluna(j);
     return produto;
   };
\end{lstlisting}
\end{frame}


\begin{frame}{\bf 3.2 Vectores}{\bf Determinante}
\begin{block}{Determinante}
Seja $A=(a_{ij})$ uma matriz $n \times n$. O determinante de $A$ é
$$ |A| = \sum_{\sigma \in S_n} (-1)^{\mathrm{sgn}(\sigma)} a_{1\sigma(1)}a_{2\sigma(2)}\cdots a_{n\sigma(n)}$$
\end{block}

$$ n!(n-1) \:\:\: \mbox{ multiplicações }$$
\end{frame}

\begin{frame}{\bf 3.2 Vectores}{\bf Determinante}

\begin{block}{Teorema de Laplace}
Para qualquer $1\leq i \leq n$ temos que $|A| = \sum_{j=1}^n a_{ij} |\widehat{A}_{ij}|.$
Onde  
$$\widehat{A}_{ij} = \left(\begin{array}{ccccc} 
{a_{11}} & \ldots &{\color{red} a_{1j}} & \ldots & {a_{1n}}\\
\vdots  & \ddots & \vdots  & \ddots & \vdots \\
{\color{red} a_{i1}} & \ldots &{\color{red} a_{ij}} & \ldots & {\color{red} a_{in}}\\  
\vdots  & \ddots & \vdots  & \ddots & \vdots \\
{a_{n1}} & \ldots &{\color{red} a_{nj}} & \ldots & { a_{nn}}\\
\end{array}\right)$$
\end{block}

$$ n!\:\:\: \mbox{ multiplicações }$$

\end{frame}


\begin{frame}{\bf 3.2 Vectores}{\bf Determinante}

\begin{block}{Eliminação de Gauss}
$$A = \left(\begin{array}{ccccc} 
{\color{red} a_{11}} & a_{12} & \ldots & {a_{1n}}\\
\vdots  & \ddots & \vdots  & \ddots & \vdots \\
{a_{i1}} & \ldots &{a_{ij}} & \ldots & {a_{in}}\\  
\vdots  & \ddots & \vdots  & \ddots & \vdots \\
{a_{n1}} & \ldots &{a_{nj}} & \ldots & { a_{nn}}\\
\end{array}\right)
\rightarrow 
A' = \left(
\begin{array}{ccccc} 
{\color{red} a_{11}} & a_{12} & \ldots & {a_{1n}}\\
\vdots  & \ddots & \vdots  & \ddots & \vdots \\
0 & \ldots &{a'_{ij}} & \ldots & {a'_{in}}\\  
0  & \ddots & \vdots  & \ddots & \vdots \\
0 & \ldots &{a'_{nj}} & \ldots & { a'_{nn}}\\
\end{array}\right)
$$

$$ \sum_{k=2}^n (k-1)k = \frac{n^3-n}{3}\:\:\: \mbox{ multiplicações }$$
\end{block}

\end{frame}



\end{document}




3. 
3.1 O template de vectores
3.2 O template de matrizes
3.3 Multiplicação rápida de matrizes: Algoritmo de Strassen.
3.4 Resolução de sistemas lineares

4. 
4.1 Introdução à geometria algébrica
4.2 Ordem nos monómios
4.3 Algoritmo da divisão para polinómios em múltiplos indeterminadas
4.4 Lemma de Dickson
4.5 Bases de Gröbner e Teorema de Hilbert
4.6 Algoritmo de Buchberger
4.7 Teoria de Eliminação
4.8 Teorema "Nullstellensatz" de Hilbert
4.9 Algoritmo de filiação de radicais
4.10 Aplicação à robótica
\end{frame}

